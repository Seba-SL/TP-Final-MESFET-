\documentclass[a4paper]{article}
\usepackage[margin=1.5cm,top=1.5cm,bottom=1.5cm,a4paper]{geometry}
\usepackage{graphicx} % Required for inserting images
\usepackage{titlesec} % Required for customizing section titles
\usepackage[margin=1.5cm,top=1.5cm, bottom=1.5cm]{geometry} % Ajusta los márgenes aquí
\usepackage{multicol} % Required for multicols environment
\usepackage{parskip}
\usepackage{etoolbox}
\usepackage{tcolorbox}
\usepackage{float}
\usepackage{amsmath}
\usepackage{dingbat}
\usepackage{booktabs} 
\usepackage{caption}
\usepackage[utf8]{inputenc}
\usepackage[T1]{fontenc}
\usepackage{amssymb}
\usepackage[spanish]{babel}

\addto\captionsspanish{
	\renewcommand{\tablename}{Tabla}
}

\usepackage{tikz}
\usetikzlibrary{arrows.meta, positioning}
\usetikzlibrary{babel}

\usepackage{csquotes}

\usepackage[
  backend=biber,
  style=apa,
  citestyle=apa
]{biblatex}

\addbibresource{referencias.bib}



%\captionsetup[table]{name = Tabla}
 
% Redefinir el comando \thesection para usar números romanos
%\renewcommand{\thesection}{\Roman{section}}
% Redefinir el comando \thesubsection para usar números romanos
%\renewcommand{\thesubsection}{\thesection.\Roman{subsection}}


\renewcommand{\thesection}{\arabic{section}}
\renewcommand{\thesubsection}{\thesection.\arabic{subsection}}
\renewcommand{\thesubsubsection}{\thesubsection.\arabic{subsubsection}}


% Personalización de la sección para que comience desde la izquierda
\titleformat{\section}[hang]{\normalfont\Large\bfseries}{\thesection}{1em}{}

\begin{document}
\setcounter{page}{1}
\begin{minipage}[h]{1\textwidth}

\title{\includegraphics[height = 1.5cm]{logofiuba.png} % Logo en la izquierda del encabezado  
\\(TB070) Dispositivos Semiconductores \\Trabajo Práctico Final : Transistor MESFET }
\author{106213 Sebastián Lazo  (slazo@fi.uba.ar) }
\date{26 de Febrero 2026}

\maketitle

 \end{minipage}

\section{Resumen}

En el presente informe se aplican los conocimientos adquiridos en la materia Dispositivos Semiconductores para el estudió y análisis del dispositivo conocido como MESFET (Metal Semiconductor Field Effect Transistor). Se abordan su estructura, parámetros característicos, aplicaciones típicas, modos de operación, así como su modelo clásico y su modelo completo. Se presentan diagramas de bandas para cada modo de operación, curvas características y se discuten los efectos no ideales que influyen en su comportamiento real. 

\section{Introducción al dispositivo}

El transistor MESFET es un dispositivo multijuntura de tres terminales: fuente, drenaje y compuerta, su estructura forma un canal por el cual fluyen los portadores desde el terminal fuente hacia el terminal de drenaje. La conductividad del canal es modulada por el potencial eléctrico aplicado en el terminal de compuerta. Puede clasificarse como una variación del transistor MOSFET, con la diferencia de que no posee una capa de óxido aislante entre la compuerta y el sustrato, característica que también comparte con el transistor JFET. Con respecto a este último, difieren en el material de la compuerta, ya que en el MESFET no es de material semiconductor, sino metálica, empleándose metales como aluminio, titanio, oro, níquel o platino.


\subsection{Motivación de su desarrollo}

Para comprender la necesidad de esta clase de transistor es necesario conocer una limitación en los transistores MOSFET, estos pueden no ser apropiados si se desea fabricar un transistor utilizando ciertos materiales semiconductores, como en el caso del arseniuro de galio (GaAs), dado que en la interfase, entre el GaAs y un aislante como el oxido, se forman un gran número de trampas superficiales que inhiben la acción compuerta. Además, las compuertas de unión no se difunden con tanta facilidad dentro del GaAs debido a la inestabilidad del material a altas temperaturas, por lo tanto no se realizan uniones P-N con facilidad. Sin embargo, dado que el GaAs tiene una mayor movilidad que el silicio para sus portadores libres, su empleo es aconsejable en aplicaciones que requieren rápidas velocidades de conmutación. La mayoría de las aplicaciones utilizan el MESFET de canal n en lugar del de canal p debido a la mayor movilidad de los portadores en los dispositivos de canal n \parencite{sze2002semiconductor}. Por consiguiente, se utilizan estructuras del tipo MESFET de canal n para cubrir esas necesidades \parencite{StreetmanBanerjee_SolidStateDevices_2006}. Otros semiconductores utilizados para fabricar MESFET's son nitruro de galio (GaN), carburo de silicio de estructura cristalina (4H-SiC) y fosforo de indio (InP), las caracteristicas de los mismos se encuentran en la tabla de la figura \ref{fig:tablapropsctransistores}. 


\begin{figure}[H]
	\centering
	\includegraphics[width=0.6\linewidth]{"../../../../Imágenes/Capturas de pantalla/TablaProp_SC_transistores"}
	\caption{Tabla con las caracteristicas princiáles de los semiconductores y el tipo de transistor fabricado con ellos \parencite{giannini2009high}.}
	\label{fig:tablapropsctransistores}
\end{figure}

\subsection{Primer prototipo}

	El MESFET fue propuesto y demostrado por primera vez por Carver Mead en 1966, se puede observar una imagen de el primer prototipo en la figura \ref{fig:primermesfet}. Poco después, Hooper y Lehrer informaron sobre su rendimiento en dispositivos de microondas en 1967 \parencite{sze2021}. Carver Mead recibió una pequeña muestra de GaAs de tipo n formado epitaxialmente (crecimiento cristalino ordenado) sobre una oblea semiaislante y comenzo a trabajar en el laboratorio grabando la capa epitaxial de su diminuta muestra de oblea, paso a paso, hasta que pudo ver la perforación desde un contacto metálico a una tensión de 10 V. El proceso que estaba usando era un grabado químico de tiempo aproximado con metanol y bromo, y aunque la tensión de 10 V era un poco alto comparado con experiencias similares, ya había pasado por varios pasos de grabado, metalización y medición (usando la relación entre capacitancia y tensión para obtener la concentración de dopaje del canal y espesor de agotamiento) para llegar a este punto. Decidió detenerse y, usando un rapidógrafo calentado (un bolígrafo de dibujo de tinta líquida de estilo antiguo) lleno de cera negra, dibujó una franja a lo largo de la oblea para definir el área activa de lo que sería la región de fuente-drenaje, y luego grabó cuidadosamente a través de la región epitaxial hasta el GaAs semiaislante en el resto. A continuación, soldó por puntos dos hojas de afeitar de filo recto a pequeñas piezas transversales para hacer una máscara para una puerta muy estrecha. Los bordes de las hojas estaban tan juntos que formaban un patrón de difracción a lo largo de sus bordes casi en contacto \parencite{Siegel2021Mead}.

\begin{figure}[H]
	\centering
	
	\begin{minipage}{0.45\textwidth}	
		
		 Usando esta máscara improvisada, evaporó un electrodo de compuerta de aluminio muy fino (de unas pocas micras de ancho) a través del hueco entre las aspas y a lo largo del punto medio de la mesa. Después, soldó contactos óhmicos de indio-mercurio a la mesa, en lados opuestos de la compuerta, para formar la fuente y el drenador. Sorprendentemente, el dispositivo funcionó y Carver pudo registrar curvas I-V del FET, dado que el transistor operaba a 10 V. Redactó el breve artículo y lo presentó a las Actas del IEEE en diciembre, denominándolo Transistor de Efecto de Campo de Puerta de Barrera Schottky, que posteriormente se convertiría en el MESFET \parencite{Siegel2021Mead}.
		
		
	\end{minipage}
	\hfill
	\begin{minipage}{0.45\textwidth}
		\centering
		\includegraphics[width=\linewidth]{Img/primer_Mesfet}
		\caption{Fotografía del primer prototipo de MESFET de GaAs, tal como está conectado a un portaobjetos de microscopio \parencite{Siegel2021Mead}.}
		\label{fig:primermesfet}
	\end{minipage}
	
\end{figure}





\subsection{Estructura}

Los MESFET's se construyen a partir de una delgada capa epitaxial de GaAs de tipo n dopada con impurezas donadoras como pueden llegar a ser silicio, azufre o selenio depositada sobre un sustrato semi-aislante tambien de GaAs pero dopado intencionalmente con cromo, que se comporta como un único aceptor cerca del centro de la banda prohibida de energía, con el objetivo de obtener un semi-aislante con una resistividad de ordenes de hasta hasta $10^9 \frac{\Omega}{cm}$ \parencite{neamen2012}. Sobre esta capa se definen las tres terminales del dispositivo: \textit{source}, \textit{drain} y \textit{gate} (fuente, drenaje y compuerta), como se aprecia en la Figura \ref{fig:estr_Belove_MESFET} y \ref{fig:estr_Sze_MESFET}.



\begin{figure}[H]
    \centering

    \begin{minipage}{0.48\textwidth}
        \centering
        \includegraphics[width=\linewidth]{Img/estr_S_M_ SZE.png}
        \caption{Estructura tridimensional del transistor MESFET \parencite{sze2021}.}
        \label{fig:estr_Belove_MESFET}
    \end{minipage}
    \hfill
    \begin{minipage}{0.48\textwidth}
        \centering
        \includegraphics[width=\linewidth]{Img/estructura_MESFET_SZE.png}
        \caption{Estructura del transistor MESFET, se observa la apertura neta del canal $b$ controlada por el ancho de agotamiento $W_D$ \parencite{sze2002semiconductor}.}
        \label{fig:estr_Sze_MESFET}
    \end{minipage}

\end{figure}



Sobre el sustrato semi-aislante se encuentra la región activa del dispositivo, conformada por la capa de GaAs dopada ligeramente, en ella se forma el canal por el cual circulan los portadores mayoritarios cuando el dispositivo está en conducción, esto se aprecia en la figura \ref{fig:neamen_strucutre}.

En las zonas correspondientes a \textit{source} y \textit{drain}, esta misma capa se dopa fuertemente con las mismas impurezas de tipo n para obtener contactos óhmicos de baja resistencia, facilitando así la inyección y recolección de portadores.

\begin{figure}[H]
    \centering

    \begin{minipage}{0.4\textwidth}
        \centering
        \includegraphics[width=\linewidth]{Img/neamen_structure.png}
        \caption{Estructura del transistor MESFET de GaAs \parencite{neamen2012}.}
       \label{fig:neamen_strucutre}
       
    \end{minipage}
    \hfill
    \begin{minipage}{0.4\textwidth}
    
        \includegraphics[width=\linewidth]{Img/estructura_microwave.png}
        \caption{Estructura del transistor MESFET \parencite{giannini2009high}.}
       \label{fig:giannini_MESFET_structure}
    \end{minipage}

\end{figure}



Finalmente, en la región correspondiente a la terminal \textit{gate}, se deposita un metal en contacto directo con el canal, formando una unión metal-semiconductor que permite controlar la conducción modulando la anchura de la región de agotamiento en el canal definida como $W_D$, esto se detalla en la figura \ref{fig:estr_Sze_MESFET}. Una diferencia crucial con respecto de los transistores de unión bipolar es que los transistores de efecto de campo no requieren de corriente de polarización y son controlados por tensión. Además, el hecho de que su funcionamiento responde a la corriente de portadores mayoritarios se los designa como transistores unipolares.



En la figura \ref{fig:giannini_MESFET_structure} se detalla su estructura en tres dimensiones, también se detalla una capa opcional de material tipo p conocido como \textit{p-buffer} con el propósito de mejorar el acople, aislando mejor el canal, reduciendo corrientes de fuga hacia el sustrato y mejorando la estabilidad del dispositivo.



El dispositivo frente a una elevada temperatura disminuye su corriente evitando un descontrol térmico, esto permite conectar fácilmente varios MESFET en paralelo, creando así un dispositivo más grande, como se detalla, por ejemplo, en la figura \ref{fig:giannini_MESFETs}. En la figura \ref{fig:mishra_microscopio} se aprecian dos imágenes reales del dispositivo mediante un microscopio, la primera mediante un corte trasversal y la segunda una visual superior. Por último, en la figura \ref{fig:esquematico_de_paper} se detalla conceptualmente la polarización usual del dispositivo.



\begin{figure}[H]
    \centering

    \begin{minipage}{0.45\textwidth}
        \centering
        \includegraphics[width=\linewidth]{Img/6W_MESFET_circuito_integrado.png}
         \caption{Imagen donde se observa un arreglo integrado de MESFET's de GaAs \parencite{giannini2009high} .}
    \label{fig:giannini_MESFETs}
   
    \includegraphics[width=\linewidth]{Img/esquematico_de_paper}
         \caption{ Esquema conceptual de polarización del MESFET \parencite{Belgat2004_MESFET_Interface}.}
    \label{fig:esquematico_de_paper}
    
	\end{minipage}
    \hfill
    \begin{minipage}{0.45\textwidth}
        \centering
        \includegraphics[width=\linewidth]{Img/Microscopio_MESFET_MISHRA.png}
         \caption{(Arriba) Una sección transversal de corte de un MESFET de 0,1 $\mu m$. (Abajo) Vista superior de la MESFET \parencite{mishra_semiconductor_2008}.}
    \label{fig:mishra_microscopio}
    \end{minipage}

\end{figure}

\subsubsection{Características del semiconductor y su dopaje}

En el caso del arseniuro de galio, es un solido semiconductor compuesto cuya red cristalina forma una estructura conocida como \textit{zincblenda} apreciable en la figura \ref{fig:zincblenda}, compuesta por arsénico del grupo V y galio del grupo III union detallada en la figura \ref{fig:red_GaAs}. Al agregar una impureza, esta reemplaza a alguno de los átomos en la red cristalina. Para obtener material tipo n, la impureza donadora debe aportar un electrón adicional respecto del átomo que reemplaza. Cuando se emplea silicio como impureza, este sustituye al galio, que posee tres electrones de valencia, mientras que el silicio posee cuatro, actuando como donador. Por otro lado, el azufre y el selenio poseen seis electrones de valencia, por lo que al reemplazar al arsénico, que posee cinco, aportan un electrón adicional, dando lugar a material tipo N \parencite{neamen2012}. 



\begin{figure}[H]
	\centering
	\begin{minipage}{0.35\textwidth}
		\centering
		\includegraphics[width=\linewidth]
		{"../../../../Imágenes/Capturas de pantalla/zincblenda"}
		\caption{Estructura cristalina tipo zincblenda del semiconductor GaAs \parencite{neamen2012}.}
		\label{fig:zincblenda}
	\end{minipage}
	\hfill
	\begin{minipage}{0.55\textwidth}
		\centering
		\includegraphics[width=0.5\linewidth]{Img/red_GaAs}
		\caption{Enlace covalente del cristal del GaAs. \parencite{BoylestadNashelsky2015}.}
		\label{fig:red_GaAs}
	\end{minipage}
\end{figure}


\subsection{Aplicaciones típicas}
	
	
	Es empleado en sistemas de comunicación por microondas, amplificadores, radiotelescopios hasta antenas parabólicas domésticas, sistemas satélitales y teléfonos celulares operando en frecuencias por encima de 3 GHz. En la figura \ref{fig:gaas-mesfet-class-ab-rfpa-schematic-diagram} se implementa un MESFET en un amplificador clase AB.
	

\begin{figure}[H]
	\centering
	\begin{minipage}[t]{0.52\textwidth}
		\vspace{0.5 cm}
		 
		 La tecnología MESFET es relevante en aplicaciones especializadas de alta potencia y alta temperatura, se suelen adoptar para frecuencias de hasta 18-20 GHz, mientras que la adopción de dispositivos de heterojunción (principalmente del tipo HEMT) se hace obligatoria para frecuencias de operación más altas \parencite{giannini2009high}.
		 
		 \vspace{1cm}
		 	
		 \begin{itemize}
		 	\item 	Radares
		 	\item 	Equipos de radiocomunicación
		 	\item 	Tacómetros
		 	\item 	Satélites
		 	\item 	Microondas
		 \end{itemize}
		 
	
	\end{minipage}
	\hfill
	\begin{minipage}[t]{0.43\textwidth}
		\vspace{0pt}
		\centering
		\includegraphics[width=1\linewidth]{../Imagenes/GaAs-MESFET-Class-AB-RFPA-schematic-diagram}
		\caption{Diagrama esquemático de un amplificador Class-AB RFPA implementando un GaAs MESFET.}
		\label{fig:gaas-mesfet-class-ab-rfpa-schematic-diagram}
		\vspace{0pt}
	\end{minipage}
\end{figure}


\subsection{Tipos de MESFET}

Al diseñar un MESFET canal n, se puede optar por una determinada concentración de impurezas y geometría del canal tal que requiera tensiones de compuerta negativas para modular o interrumpir la conducción, el cual se conoce como modo empobrecimiento (D-MESFET o Depletion-mode MESFET) o tensiones de control positivas, este ultimo nombrado como modo enriquecimiento (E-MESFET o Enhancement-mode MESFET).


\begin{figure}[H]
	\centering
	\begin{minipage}[t]{0.3\textwidth}
		\begin{figure}[H]
			\centering
			\includegraphics[width=0.85\linewidth]{Img/Tipos_MESFET}
			\caption{ Simbología típica de cada MESFET de canal n.}
			\label{fig:tiposmesfet}
		\end{figure}
		
	\end{minipage}
	\hfill
	\begin{minipage}[t]{0.65\textwidth}
	De esta forma se pueden clasificar en dos tipos de MESFET por cada tipo de canal, los mismos comúnmente se diferencian con ayuda de los símbolos que se presentan en la figura \ref{fig:tiposmesfet}.
	Es importante tener en cuenta, sin embargo, que el canal debe ser de material tipo n en un MESFET. La movilidad de los huecos en GaAs es relativamente baja comparada con la de portadores de carga negativa por lo que se pierde la ventaja de utilizar GaAs en aplicaciones de alta velocidad. El resultado es:
	Los MESFET tipo empobrecimiento y tipo enriquecimiento se hacen con un canal n entre el drenaje y la fuente y, por consiguiente, sólo los MESFET tipo n son comerciales \parencite{BoylestadNashelsky2015}.
	
	\end{minipage}
\end{figure}




\section{Parámetros Característicos}

En el informe se tomara como objeto de estudio un transistor MESFET formado por GaAs de canal n de empobrecimiento (D-MESFET), con silicio como impureza dopante en el canal y titanio como el metal de la compuerta a una temperatura de 300 K. A continuación se presentaran los diferentes parámetros típicos de referencia utilizados en el informe.


\subsection{Parámetros físicos}

Los materiales que forman el dispositivo desprenden diferentes características descriptas por los siguientes parámetros (\ref{tab:param_fis_mesfet}), tales determinan propiedades fundamentales del dispositivo, como la formación de la barrera \textit{Schottky}, la zona de vaciamiento, el transporte de portadores, etc.


\begin{table}[H]
	\centering
	\caption{Parámetros físicos característicos de los materiales utilizados en el análisis del MESFET}
	\label{tab:param_fis_mesfet}
	\begin{tabular}{|l|c|c|c|c|}
		\hline
		\textbf{Parámetro} & \textbf{GaAs} & \textbf{Si (dopante)} & \textbf{Ti} & \textbf{Cr} \\ \hline
		Tipo de material & Semiconductor & Semiconductor & Metal & Metal \\ \hline
		Energía de banda prohibida $E_g$ [eV] & 1.42 & 1.12 & -- & -- \\ \hline
		Constante dieléctrica relativa $\epsilon_s$ & $12.9\cdot \epsilon_0$ & $11.7\cdot \epsilon_0$ & -- & -- \\ \hline
		Afinidad electrónica $\chi$ [eV] & 4.07 & -- & -- & -- \\ \hline
		Función trabajo $\Phi_M$ [eV] & -- & -- & 4.33 & 4.50 \\ \hline
		Movilidad electrónica $\mu_n$ [cm$^2$/Vs] & 8500 & -- & -- & -- \\ \hline
		Campo critico $\xi_{max}$ [$\frac{kV}{cm}$] & $300$ & $300$ & -- & -- \\ \hline
		Masa efectiva ${m}^*_n$ &$0,067\cdot m_o$ & -- & -- & -- \\ \hline
		Masa efectiva ${m}^*_p$  &$0,47\cdot m_o$& -- & -- & -- \\ \hline
		Concentración intrínseca de portadores $n_i$ [cm$^{-3}$] & $1,79 \cdot 10^{6}$ &  $1,45 \cdot 10^{10}$ & -- & -- \\ \hline
		Concentración  $N_D$ [cm$^{-3}$] & $4 \cdot 10 ^{15}$ & -- & -- & -- \\ \hline
		Concentración  $N_A$ [cm$^{-3}$] & 0 &--& -- & -- \\ \hline
		Comportamiento en el MESFET & Canal & Dopaje & Schottky / Óhmico & Aislante \\ \hline
	\end{tabular}
\end{table}

Siendo $m_o$ = $9,109 \cdot 10^{-31} \,Kg $.

\subsection{Parámetros Geométricos}

En la siguiente tabla se presentan los parámetros que describen las dimensiones geométricas del dispositivo (\ref{tab:param_geom_mesfet}). 

\begin{table}[H]
	\centering
	\caption{Parámetros geométricos característicos utilizados en el análisis del MESFET}
	\label{tab:param_geom_mesfet}
	\begin{tabular}{|l|c|c|l|}
		\hline
		\textbf{Parámetro} & \textbf{Símbolo} & \textbf{Valor} & \textbf{Descripción} \\ \hline
		Longitud de la compuerta 
		& $L$ 
		& $2 \,\mu m$
		& Longitud del canal en la dirección $x$ (source--drain) \\ \hline
		
		Ancho del dispositivo 
		& $Z$ 
		& $10 \,\mu m$
		& Dimensión del canal en la dirección $z$ (ancho del dispositivo) \\ \hline
		
		Espesor del canal 
		& $a$ 
		& $1,2 \,\mu m$
		& Altura física del canal en la dirección $y$ \\ \hline
		
		Ancho de la zona de vaciamiento 
		& $W_d$ 
		& -
		& Extensión de la región de vaciamiento bajo la compuerta \\ \hline
		
		Espesor efectivo del canal 
		& $d_{\text{eff}}$ 
		& -
		& Espesor del canal conductor: $d_{\text{eff}} = a - W_d$ \\ \hline
		
		Área efectiva del canal 
		& $A_{\text{ch}}$ 
		& -
		& Área transversal del canal: $A_{\text{ch}} = Z \cdot d_{\text{eff}}$ \\ \hline
		
	
	\end{tabular}
\end{table}



\subsection{Parámetros eléctricos}

El comportamiento eléctrico del dispositivo se podrá describir con ayuda de las definiciónes de diferentes tensiones, corrientes, resistencias, capacitancias y otros parámetros eléctricos para facilitar el análisis del dispositivo (\ref{tab:param_elec_mesfet}). 

\begin{table}[H]
	\centering
	\caption{Parámetros eléctricos característicos utilizados en el análisis del MESFET}
	\label{tab:param_elec_mesfet}
	\begin{tabular}{|l|c|c|l|}
		\hline
		\textbf{Parámetro} & \textbf{Símbolo} & \textbf{Unidad} & \textbf{Descripción} \\ \hline
		Tensión compuerta--fuente & $V_{GS}$ & V & Tensión aplicada a la unión Schottky de la compuerta.\\ \hline
		Tensión drenaje--fuente & $V_{DS}$ & V & Tensión aplicada entre drenaje y fuente. \\ \hline
		Corriente de drenaje & $I_D$ & A & Corriente que circula por el canal del MESFET. \\ \hline
		Corriente de compuerta & $I_G$ & A & Corriente de fuga de la unión Schottky.\\ \hline
		Tensión de umbral (pinch--off) & $V_P$ & V & Tensión $V_{GS}$ para la cual el canal se estrangula por completo. \\ \hline
		Corriente de saturación & $I_{DSS}$ & A & Corriente de drenaje para $V_{GS} = 0$ (D-MESFET).\\ \hline
	\end{tabular}
\end{table}


\subsection{Parámetros de rendimiento}

Para un posterior análisis de funcionamiento detallado contemplando posibles condiciones no ideales, es necesario detallar parámetros de rendimiento como los descriptos por la siguiente tabla (\ref{tab:param_rend_mesfet}). 



\begin{table}[H]
	\centering
	\caption{Parámetros de rendimiento característicos del transistor MESFET}
	\label{tab:param_rend_mesfet}
	\begin{tabular}{|l|c|l|}
		\hline
		\textbf{Parámetro} & \textbf{Símbolo} & \textbf{Descripción} \\ \hline
		Ancho de banda útil & $BW$ & Rango de frecuencias de operación efectiva. \\ \hline
		Factor de ruido & $NF$ & Degradación de la relación señal--ruido introducida por el dispositivo. \\ \hline
		Temperatura de operación & $T_{op}$ & Rango de temperatura en condiciones nominales. \\ \hline
		Rango de tensión compuerta--fuente & $V_{GS}$ & Intervalo de polarización segura de la compuerta. \\ \hline
		Rango de tensión drenaje--fuente & $V_{DS}$ & Intervalo de operación sin ruptura. \\ \hline
		Rango de corriente de drenaje & $I_D$ & Corriente admisible en régimen continuo. \\ \hline
		Región de operación recomendada & -- & Lineal, saturación o corte. \\ \hline
		Potencia máxima disipada  & $P_{max}$ & Potencia máxima admisible sin degradación. \\ \hline
	\end{tabular}
\end{table}



\section{Principio de Funcionamiento}

Una unión metal–semiconductor puede dar lugar a dos tipos de contacto, dependiendo de la relación entre las funciones trabajo de los materiales que la conforman y del nivel de dopaje del semiconductor. El contacto óhmico, presente en los terminales \textit{drain} y \textit{source}, se obtiene utilizando un semiconductor tipo n fuertemente dopado, lo que permite una baja resistencia de contacto y condiciones cercanas a $\phi_m < \phi_{SC}$.

El segundo caso corresponde a la formación de una unión rectificante del tipo \textit{Schottky}, para la cual rige la relación $\phi_m > \phi_{SC}$. Esta situación se presenta cuando el semiconductor tipo n se encuentra levemente dopado y es la empleada en el terminal \textit{gate}. Este tipo de unión es también característica en los diodos rectificadores de rápida conmutación.

Al igual que en los diodos Schottky polarizados en inversa, esta unión da lugar a la formación de una región de vaciamiento de portadores libres en el semiconductor. En el transistor MESFET, la extensión de dicha región es modulada mediante la tensión aplicada entre \textit{gate} y \textit{source}, permitiendo el estrangulamiento o la apertura del canal conductor. De esta forma, se controla la corriente que circula entre \textit{drain} y \textit{source}, constituyendo este mecanismo el principio fundamental de funcionamiento del dispositivo.


\section{Operación básica}

El dispositivo se opera generalmente como conmutador o como parte de un circuito amplificador de señal. A partir de las referencias de tensión y corriente inspiradas en la figura \ref{fig:simbolomasesquema}, se introduce a continuación el principio de operación del MESFET, mientras que en las secciones siguientes se desarrollará su funcionamiento en mayor detalle.

Para el análisis del comportamiento del dispositivo se utilizará una nomenclatura que permita distinguir entre señales continuas y señales alternas en el tiempo. De este modo, cualquier magnitud eléctrica podrá descomponerse en una componente continua (DC) y una componente alterna (AC). 

Como ejemplo, la tensión entre \textit{gate} y \textit{source} puede expresarse como la suma de una componente continua, denotada mediante notación mayúscula \( V_{GS} \), y una componente alterna, representada en minúscula \( v_{gs} \). La señal total dependiente del tiempo resulta de la superposición de ambas componentes.

\begin{itemize}
	\item Tensión entre \textit{gate} y \textit{source}:\hspace{0.5cm} $ \  v_{GS} =  V_{GS} + v_{gs} $
	\item Tensión entre \textit{drain} y \textit{source}:\hspace{0.5cm} $   v_{DS} =  V_{DS} + v_{ds} $
	\item Corriente en rama \textit{drain}:\hspace{1cm} $   i_{D} =  I_{D} + i_{d} $
\end{itemize}


\begin{figure}[H]
	\centering
	\includegraphics[width=0.7\linewidth]{Img/simbolo_mas_esquema}
	\caption{Simbología tipica y esquema \parencite{Sharma_Metal_Semiconductor}. }
	\label{fig:simbolomasesquema}
\end{figure}





\subsection{Polarización}

La polarización del transistor forma parte de la operación del mismo, consiste en establecer las condiciones de tensión y corriente necesarias para fijar un punto de operación, comúnmente denominado punto $Q$, definido por el conjunto de valores $\{ V_{GS}, V_{DS}, I_D \}$. Dicho punto de operación determina la región de funcionamiento del dispositivo y condiciona su comportamiento frente a pequeñas variaciones de señal.

La elección del punto $Q$ depende de la aplicación específica del MESFET, ya sea como conmutador o como amplificador, y se realiza de manera tal que el dispositivo opere en una región adecuada de sus características estáticas.


\subsection{Modo corte}

El modo corte sucede cuando la tensión $V_{GS}$ no supera la tensión $V_p$ o tensión umbral, y el transistor idealmente no tendra corriente a travez de la rama del \textit{drain}, tal como indican las expresiones (\ref{equation:modo_corte}).

\begin{equation}\label{equation:modo_corte}
		V_{GS} \le V_P \quad \Rightarrow \quad I_D = 0
\end{equation}

\subsection{Modo en estrangulación}

El modo estrangulación es el rango en el cual la corriente de \textit{drain} se puede controlar mediante la tensión $V_{GS}$, es buscado en aplicaciones de amplificación de señal. En la expresión (\ref{equation:modo_estrangulacion}) se encuentran las condiciones de polarización y la corriente $I_D$ en función de $V_{GS}$ obtenida según el modelo de Shockley, que utiliza un valor experimental de $I_{DSS} $.


\begin{equation}\label{equation:modo_estrangulacion}
		\begin{cases}
			V_{GS} > V_P \\
			V_{DS} \ge V_{GS} - V_P
		\end{cases} \quad
		I_D = I_{DSS}
		\left(
		1 - \frac{V_{GS}}{V_P}
		\right)^2
\end{equation}



\subsection{Modo óhmico}


El modo lineal o óhmico responde linealmente tal como ocurriría con un resistor y su corriente de \textit{drain} se rige por las condiciones de la expresión (\ref{equation: modo_lineal}) para el modelo clásico de Shockley.
 
\begin{equation}\label{equation: modo_lineal}
	\begin{aligned}
		\text{para } \quad
		\begin{cases}
			V_{GS} > V_P \\
			0 \le V_{DS} < V_{GS} - V_P
		\end{cases}
		\qquad
		I_D &=
		\frac{2 I_{DSS}}{V_P^2}
		\left[
		\left( V_{GS} - V_P \right) V_{DS}
		- \frac{V_{DS}^2}{2}
		\right]
	\end{aligned}
\end{equation}

\section{Diagrama de bandas}

El análisis en profundidad parte de las bandas de energía de los materiales que conforman el dispositivo, como se comportan en las interfases entre ellos, en equilibrio termodinámico y bajo potenciales externos.

En la física del estado sólido, los átomos se modelan mediante niveles de energía en donde es probable encontrar sus electrones, pero cuando estos átomos forman redes cristalinas periódicas, como en el caso de los solidos usados en dispositivos semiconductores, los niveles mas alejados del núcleo de cada átomo individual se solapan con los del resto formando el concepto de bandas de energía con el cual se estudiaran los fenómenos físicos de los dispositivos. Se definen dos bandas características en el caso de los semiconductores y aislantes separadas por una banda prohibida (\textit{bandgap}), donde en principio idealmente no hay niveles de energía disponibles para portadores libres. La banda donde los electrones están más ligados al sólido es la banda de valencia, esta se encuentra normalmente llena de electrones y separada por la banda prohibida se encuentra la banda de conducción, normalmente vacía. Para esta sección se hará uso de los valores indicados en la tabla \ref{tab:param_fis_mesfet}.


\begin{itemize}
	\item $E_o$ : Energía de un electrón libre en el vacío.
	\item $E_f$ : Energía de Fermi (nivel de energía con probabilidad de ocupación $\frac{1}{2}$ en equilibrio térmico).
	\item $\chi$ = $E_o$ - $E_c$: Afinidad electrónica, energía necesaria para llevar un electrón desde el borde inferior de la banda de conducción hasta el vacío.
	\item $\phi = E_o - E_f$ : Función trabajo 
	\item $ E_v$ : Nivel de energía superior de la banda de valencia, usualmente usado de referencia.
	\item $ E_c$ : Nivel de energía inferior de la banda de conducción.
	\item $E_g = E_c - E_v$ : La energía de \textit{gap} es la diferencia entre la banda de valencia y de conducción.
\end{itemize}


Existen tres clasificaciones principales para los solidos: Aislantes, semiconductores y conductores, y sus diagramas de bandas típicos se encuentran en la figura \ref{fig:bandasindividuales}, se puede observar como en un aislante la diferencia de energía entre la banda de valencia y de conducción es considerablemente mayor que en la de un semiconductor, mientras que en el caso del conductor ambas bandas se encuentran solapadas siendo indistinguible un $E_{g}$.

\begin{figure}[H]
	\centering
	\includegraphics[width=1\linewidth]{Img/Bandas_individuales}
	\caption{ Diagramas de bandas de los tres principales tipos de solidos \parencite{boylestad2013}.}
	\label{fig:bandasindividuales}
\end{figure}

Cuando un solido no es dopado, se dice que es un material intrínseco, al añadirle impurezas el diagrama de bandas de este solido se modifica de acuerdo a la concentración y el tipo de estas impurezas. 

Como ya se menciono el MESFET se forma principalmente con metal (conductor), semiconductor dopado tipo n, y un semi-aislante (Semiconductor GaAs con cromo como impureza). Al doparse el semiconductor, se forma un nivel donor y/o aceptor dentro de la banda prohibida de acuerdo al tipo de dopaje.

\subsection{Portadores}

Un electrón excitado (por ejemplo, mediante agitación térmica) dentro de la banda de conducción es una carga libre negativa que se mueve a través de todo el cristal bajo la influencia de gradientes de temperatura o potencial, normalmente denotada como \textit{n}. La deficiencia de carga de valencia producida en la banda de conducción por la excitación de un electrón es su contratarte como carga positiva, denominada hueco o \textit{p}.



\subsection{Distribución de energía en un solido }

La distribución más probable de energía de un conjunto de portadores libres en equilibrio térmico, sujetos al principio de exclusión de Pauli que establece que no pueden existir dos electrones que tengan un conjunto de números cuánticos idéntico, es la distribución de Fermi. La probabilidad $f_n$ de que un nivel de energía E, esté ocupado en equilibrio térmico por un electrón esta dada por la función de Fermi-Dirac.
\\
\begin{equation}
	f_n(E) = \frac{1}{1 + e^{\frac{(E- E_f)}{kT}}}
\end{equation}

Donde $k = 8,617 \cdot 10^{-5} \, \frac{qV}{K}$ es la constante de Boltzmann y $E_f$ se conoce como nivel de Fermi, es el nivel de energía a la cual un estado cuántico tiene una probabilidad de $f_n(E_f) = \frac{1}{2}$.


Mientras que la probabilidad $f_p$ de que un nivel E se encuentre ocupado por un hueco se define como $f_p = 1 - f_n$.

\subsubsection{Energía de Fermi intrínseca}

Para un semiconductor intrínseco, es decir sin impurezas, se puede llegar a conocer su energía de Fermi intrínseca utilizando la expresión (\ref{equation:E_fi}) que la relaciona con las masas efectivas de los portadores libres.

\begin{equation}\label{equation:E_fi}
	E_{i} = \frac{1}{2}\left(E_g\right) + \frac{3}{4} k T \cdot \ln\left(\frac{m_p^{*}}{m_n^{*}}\right)
\end{equation}

Cuando el semiconductor tiene impurezas ya sean de tipo p o de tipo n, su nivel de Fermi se ya no sera el intrínseco, seguirá la relación en función de las concentraciones de átomos donadores $N_D$ o aceptadores $N_A$ como indica la expresión \ref{equation:E_f}, donde se asumirá que el total de impurezas aporto portadores libres: 

\begin{equation}\label{equation:E_f}
	E_f =
	\begin{cases}
		E_{i} + kT \ln\!\left(\dfrac{N_D - N_A}{n_i}\right), & \text{si } N_D > N_A \\[6pt]
		E_{i} - kT \ln\!\left(\dfrac{N_D - N_A}{n_i}\right), & \text{si } N_D < N_A
	\end{cases}
\end{equation}

\subsection{Diagramas de banda aislados}
A continuación se presentan los diagramas de los distintos materiales que componen las junturas del transistor de forma aislada.

\subsubsection{Diagrama de banda del conductor: Titanio}

\begin{figure}[H]
% --------- Texto (izquierda) ----------
\begin{minipage}[c]{0.45\textwidth}

El titanio no tiene brecha entre bandas de conducción y valencia ya que se solapan (no se distingue su $E_g$), esto permite el movimiento de electrones libres entre ambas bandas obteniendo alta conductividad. La función trabajo ($\Phi_M $) es la energía mínima, medida en electronvoltios (\(eV\)), necesaria para extraer un electrón desde el nivel de Fermi (\(E_{F}\)) del interior del titanio al vacío justo fuera de su superficie. 

\begin{equation}
	\Phi_M = 4,33 \; eV 
\end{equation}

\end{minipage}
\hfill
% --------- Gráfico (derecha) ----------
\begin{minipage}[c]{0.5\textwidth}
	\centering
	\begin{tikzpicture}[
		>=Latex,
		axis/.style={->, thick},
		band/.style={thick},
		level/.style={dashed, thick},
		annot/.style={font=\small}
		]
		
	
	
		% Banda metálica (ocupada)
		\fill[gray!40] (-3,-2) rectangle (2.5,0);
		\node[annot] at (0,-1) {Banda de Valencia};
		
		% Banda metálica (desocupada)
		\fill[gray!20] (-3,0) rectangle (2.5,2.165);
		\node[annot] at (0,0.6) {Banda de Conducción};
		
			% Ejes
		\draw[axis] (-3,-3) -- (-3,3) node[above] {Energía};
		% Ejes
		\draw[axis] (-3,-2) -- (3,-2) node[right] {Posición};
		
		
		
		% Nivel de Fermi
		\draw[level] (-3,0) -- (3,0);
		\node[annot, right] at (3.2,0) {$E_f$};
		
		% Nivel de vacío
		\draw[level] (-3,2.165) -- (3,2.165);
		\node[annot, right] at (3.5,2) {$E_0$};
		
		
		
		% Función trabajo
		\draw[<->, thick] (-2.2,0) -- (-2.2,2.165);
		\node[annot, left] at (-1.2,1) {$q\Phi_M$};
		
		% Etiqueta material
		\node[annot] at (0,-4) {Metal
			: Titanio (Ti)};
		
	\end{tikzpicture}
	
\end{minipage}

\caption{Diagrama de banda del conductor, en el caso de estudio titanio.}
\end{figure}




\subsubsection{Semiconductor dopado tipo n: GaAs-Si}

\begin{figure}[H]
	% --------- Texto (izquierda) ----------
	\begin{minipage}[c]{0.45\textwidth}
		
		El semiconductor dopado con impurezas tipo n, formara un nivel cercano e $E_c$ en la banda prohibida conocido como nivel donor, también el nivel de Fermi en equilibrio se colocara en una posición mas cercana a la banda de conducción a comparación del intrínseco (\ref{equation:nivel_fermi_intrinseca}). Se puede hallar el nivel de Fermi por medio de la relación (\ref{equation:nivel_fermi_tipon}) tomando como referencia $E_v = 0$ y se cumplira con el metal la relación $\phi_M > \phi_{SC_{n}}$ :
		
		\begin{equation}\label{equation:nivel_fermi_intrinseca}
			E_i =  \frac{1}{2}\left(E_g\right) + \frac{3}{4} k T \cdot \ln\left(\frac{m_p^{*}}{m_n^{*}}\right) = 0,71 \; eV
		\end{equation}
		
		\begin{equation}\label{equation:nivel_fermi_tipon}
			E_f = \frac{1,42}{2}\, eV + kT\cdot \ln\!\left(\frac{4\cdot 10^{15}}{1.79\cdot 10^{6}}\right) = 1,25 \; eV
		\end{equation}
	
		Para el caso del GaAs fuertemente dopado para las uniones óhmicas de los terminales \textit{drain} y \textit{source} el diagrama sera similar pero con un nivel de Fermi mayor, lo que dara una relación con el metal de $\phi_M < \phi_{SC_{n^+}}$. 
		
		 
	\end{minipage}
	\hfill
	% --------- Gráfico (derecha) ----------
	\begin{minipage}[c]{0.5\textwidth}
		\centering
		\begin{tikzpicture}[
			>=Latex,
			axis/.style={->, thick},
			band/.style={thick},
			level/.style={dashed, thick},
			annot/.style={font=\small}
			]
			
			
			
			% Banda metálica (ocupada)
			\fill[gray!40](-3,-2) rectangle (2.5,-1.11);
			\node[annot] at (0,-1.5) {Banda de Valencia};
			
			% Banda metálica (desocupada)
			\fill[gray!20] (-3,1) rectangle (2.5,2);
			\node[annot] at (0,1.5) {Banda de Conducción};
			
			% Ejes
			\draw[axis] (-3,-3) -- (-3,3) node[above] {Energía};
			% Ejes
			\draw[axis] (-3,-2) -- (3,-2) node[right] {Posición};
			
			
			% Nivel de vacío
			\draw[level] (-3,2) -- (3,2);
			\node[annot, right] at (3.5,2) {$E_0$};
				
			% Nivel de Fermi (tipo N, cercano a Ec)
			\draw[level] (-3,0.7) -- (2.5,0.7);
			\node[annot, right] at (2.5,0.7) {$E_f$};
			
			% Nivel de Fermi intrinseco
			\draw[level] (-3,-0.1) -- (2.5,-0.1);
			\node[annot, right] at (2.5,-0.1) {$E_i$};
			
			
			% Gap
			\draw[<->, thick] (-2.1,-1.2) -- (-2.1,1.0);
			\node[annot, left] at (-2.1,-0.5) {$E_g$};
			
			% Afinidad electrónica
			\draw[<->, thick] (2,1) -- (2,2);
			\node[annot, right] at (2.1,1.65) {$q\chi$};
			
			% Etiqueta material
			\node[annot] at (0,-4) {Semiconductor: GaAs tipo N};
		\end{tikzpicture}
		
	\end{minipage}
	
	\caption{Diagrama de bandas del semiconductor GaAs tipo n levemente dopado.}
\end{figure}




%\subsubsection{Semi-aislante: GaAs-Cr}


%\begin{figure}[H]
%	% --------- Texto (izquierda) ----------
%	\begin{minipage}[c]{0.45\textwidth}
%		Para lograr un semiconductor actuando como semi aislante se le añade como impurezas un metal, en este caso cromo, que forma un nivel en la banda prohibida que recolecta los portadores libres impidiendo que se formen, este nivel se conoce como nivel profundo (\textit{deep level}) que se forma cerca del centro energético, si un electrón llega a la banda de conducción, puede ser capturado por el nivel profundo del Cr, quedando localizado (no contribuye a conducción).
%	\end{minipage}
%	\hfill
	% --------- Gráfico (derecha) ----------
%	\begin{minipage}[c]{0.5\textwidth}
%		\centering
%		\begin{tikzpicture}[
%			>=Latex,
%			axis/.style={->, thick},
%			band/.style={thick},
%			level/.style={dashed, thick},
%			annot/.style={font=\small}
%			]
			
			
			
			% Banda metálica (ocupada)
%			\fill[gray!40](-3,-2) rectangle (2.5,-1.4);
%			\node[annot] at (0,-1.8) {Banda de Valencia};
			
			% Banda metálica (desocupada)
	%		\fill[gray!20] (-3,1.2) rectangle (2.6,2.6);
	%		\node[annot] at (0,1.5) {Banda de Conducción};
			
			% Ejes
	%		\draw[axis] (-3,-3) -- (-3,3) node[above] {Energía};
			% Ejes
	%		\draw[axis] (-3,-2) -- (3,-2) node[right] {Posición};
				% Nivel de vacío
	%		\draw[level] (-3,2.6) -- (2.5,2.6);
	%		\node[annot, right] at (2.5,2.6) {$E_0$};
			
			% Banda de conducción
	%		\draw[band] (-3,1.2) -- (2.5,1.2);
	%		\node[annot, right] at (2.5,1.2) {$E_c$};
			
			% Banda de valencia
	%		\draw[band] (-3,-1.4) -- (2.5,-1.4);
	%		\node[annot, right] at (2.5,-1.4) {$E_v$};
			
			% Nivel profundo del Cr
	%		\draw[band] (-3,-0.3) -- (2.5,-0.3);
	%		\node[annot, right] at (2.5,-0.45) {$E_{Cr}$};
			
			% Nivel de Fermi (casi centrado en el gap)
	%		\draw[level] (-3,-0.2) -- (3.4,-0.2);
	%		\node[annot, right] at (3.5,-0.2) {$E_i$};
			
			% Gap
	%		\draw[<->, thick] (-2.1,-1.4) -- (-2.1,1.2);
	%		\node[annot, left] at (-2.1,0.5) {$E_g$};
			
			% Afinidad electrónica
	%		\draw[<->, thick] (2.1,1.2) -- (2.1,2.6);
	%		\node[annot, right] at (2.1,1.9) {$q\chi$};
			
			% Etiqueta material
	%		\node[annot] at (0,-4) {Semiconductor: GaAs semi-aislante (Cr)};
	%	\end{tikzpicture}
		
	%\end{minipage}
	
	%\caption{Diagrama de bandas del sustrato semi-aislante de GaAs dopado con Cr.}
%\end{figure}






\subsection{Diagrama de bandas en equilibrio}


\textbf{Si dos o más sistemas en ponen contacto y alcanzan el ETD  sus $E_f$ deben ser iguales.}

En equilibrio termodinámico la juntura Schottky, formada por el metal-semiconductor ($\Phi_M > \Phi_S$), se comporta formando en tres regiones:

\begin{itemize}
	\item Región plana del M (en ETD el metal no soporta carga en volumen).
	\item Región de flexión en el SC.
	\item Región plana en el SC.
\end{itemize}

Se definen la barrera de potencial entre el metal y el semiconductor $\Phi_B$ = ($\Phi_M - \chi$) y barrera de juntura $q\cdot \Phi_{bi}$ = $q\cdot \Phi_B -  (E_c - E_f)$, también se puede apreciar como la flexión determina la zona de vaciamiento $W_d$ en equilibrio definida como $W_{d0}$. A su vez, se obtiene la tensión de juntura : $ V_{bi} = (\Phi_{M} - \chi_{sc} ) - ( \frac{E_c - E_f}{q}) $. 


Unión de barrera Schottky:
\begin{figure}[H]
\begin{tikzpicture}
	[
	>=Latex,
	axis/.style={->, thick},
	band/.style={thick},
	level/.style={dashed, thick},
	annot/.style={font=\small}
	]

	% --------------------
	% BANDA DE CONDUCCIÓN
	% --------------------
	\fill[gray!20]
	(3,2.0)
	.. controls (3.2,2.0) and (4.4,1.2) ..
	(6,1.2)
	-- (10,1.2)
	-- (10,0.0)
	-- (6,0.0)
	.. controls (4.4,0.0) and (3.2,0.6) ..
	(3,0.6)
	-- cycle;
	
	\draw[band]
	(3,0.6)
	.. controls (3.2,0.6) and (4.4,0.0) ..
	(6,0.0)
	-- (10,0.0);
	
	\node[annot] at (10.3,0) {$E_C$};
	
	% --------------------
	% NIVEL DE FERMI
	% --------------------
	\draw[level] (-3,-0.8) -- (10.3,-0.8);
	\node[annot, right] at (10.5,-0.8) {$E_F$};
	
	% --------------------
	% BANDA DE VALENCIA (MÁS ABAJO)
	% --------------------
	\fill[gray!40]
	(3,-4)
	-- (3,-1.6)
	.. controls (3.2,-1.6) and (4.4,-2.3) ..
	(6,-2.3)
	-- (10,-2.3)
	-- (10,-4)
	-- cycle;
	
	\draw[band]
	(3,-1.6)
	.. controls (3.2,-1.6) and (4.4,-2.3) ..
	(6,-2.3)
	-- (10,-2.3);
	
	\node[annot, right] at (10.3,-2.3) {$E_V$};
	
	% --------------------
	% INTERFAZ
	% --------------------
	\draw[thick] (3,-4.2) -- (3,2.4);
	\node[annot, above] at (3,2.4) {Interfaz};
	\node[annot, below] at (3,-4.2) {$y=0$};
	
	% --------------------
	% ALTURA DE BARRERA SCHOTTKY ΦB
	% --------------------
	\draw[level] (3,0.6) -- (6.5,0.6);
	\draw[<->, thick] (2.8,0.6) -- (2.8,-0.8);
	\node[annot, left] at (2.7,0) {$q\cdot \Phi_B$};
	
	
		% --------------------
	% \phi_X
	% --------------------

	\draw[<->, thick] (9.5,0) -- (9.5,1.2);
	\node[annot, left] at (9.4,0.5) {$q\cdot \chi$};
	
	% -----------------
	
	% --------------------
	% POTENCIAL INTERNO φ_bi
	% --------------------
	\draw[<->, thick] (6.5,0.0) -- (6.5,0.6);
	\node[annot, right] at (6.6,0.3) {$q \cdot \phi_{bi}$};
	
	% --------------------
	% ANCHO DE AGOTAMIENTO W
	% --------------------
	
	\draw[level] (6,-4.2) -- (6,2.4);

	\draw[<->, thick] (3,-3.2) -- (6,-3.2);
	\node[annot, below] at (4.5,-3.2) {$W_{d0}$};
	
	% --------------------
	% FUNCIÓN TRABAJO METAL
	% --------------------
	\draw[<->, thick] (-2.4,-0.8) -- (-2.4,2.0);
	\node[annot,left] at (-1.3,0.6) {$q \cdot \Phi_M$};
	
	
		% --------------------
	% FUNCIÓN TRABAJO GaAs
	% --------------------
	\draw[<->, thick] (8,-0.8) -- (8,1.2);
	\node[annot,left] at (9,-0.4) {$q\cdot \Phi_S$};
	
	
	% --------------------
	% EJES
	% --------------------
	\draw[axis] (-3,-4) -- (-3,2.4) node[above] {Energía};
	\draw[axis] (-3,-4) -- (10.4,-4) node[right] {Posición y};
	
		
	% --------------------
	% NIVEL DE VACÍO E0
	% --------------------
	\draw[thick] (-3,2.0) -- (3,2.0);
	\draw[band]
	(3,2.0)
	.. controls (3.2,2.0) and (4.4,1.2) ..
	(6,1.2)
	-- (10.3,1.2);
	\node[annot, right] at (10.4,1.2) {$E_0$};
	
	
	% --------------------
	% MATERIALES
	% --------------------
	\node[annot] at (-1,-4.6) {Metal: Titanio (Ti)};
	\node[annot] at (7,-4.6) {Semiconductor: GaAs tipo N};
	
\end{tikzpicture}

\end{figure}


Unión de contacto óhmico: Para el caso donde $\Phi_m < \Phi_{s} $ (tipo n), al poner en contacto ambos materiales, hay un flujo de electrones  desde el metal hacia el semiconductor. Siendo el semiconductor mas dopado cercano al contacto.

\begin{figure}[H]
	\begin{tikzpicture}
		[
		>=Latex,
		axis/.style={->, thick},
		band/.style={thick},
		level/.style={dashed, thick},
		annot/.style={font=\small}
		]
		
		% --------------------
		% BANDA DE CONDUCCIÓN
		% --------------------
		\fill[gray!20]
		(3,1.165)
		.. controls (3.2,1.165) and (4.4,1.8) ..
		(6,1.8)
		-- (10,1.8)
		-- (10,0.8)
		-- (6,0.8)
		.. controls (4.4,0.8) and (3.2,0.165) ..
		(3,0.165)
		-- cycle;
		
		\draw[band]
		(3,0.165)
		.. controls (3.2,0.165) and (4.4,0.8) ..
		(6,0.8)
		-- (10,0.8);
		
		\node[annot] at (10.3,0.8) {$E_C$};
		
		% --------------------
		% NIVEL DE FERMI
		% --------------------
		\draw[level] (-3,0) -- (10.3,0);
		\node[annot, right] at (10.5,0) {$E_F$};
		
		% --------------------
		% BANDA DE VALENCIA (MÁS ABAJO)
		% --------------------
		\fill[gray!40]
		(3,-3)
		-- (3,-2.135)
		.. controls (3.2,-2.135) and (4.4,-1.5) ..
		(6,-1.5)
		-- (10,-1.5)
		-- (10,-3)
		-- cycle;
		
		\draw[band]
		(3,-2.135)
		.. controls (3.2,-2.135) and (4.4,-1.5) ..
		(6,-1.5)
		-- (10,-1.5);
		
		\node[annot, right] at (10.3,-1.5) {$E_V$};
		
		% --------------------
		% INTERFAZ
		% --------------------
		\draw[thick] (3,-3.2) -- (3,2.4);
		\node[annot, above] at (3,2.4) {Interfaz};
		\node[annot, below] at (3,-3.2) {$y=0$};
		
		% --------------------
		% ALTURA DE BARRERA SCHOTTKY ΦB
		% --------------------
		\draw[level] (3,1.8) -- (6.5,1.8);
		\draw[<->, thick] (2.8,1.165) -- (2.8,1.8);
		\node[annot, left] at (2.7,1.8) {$q\cdot \Phi_B$};
		
		
		% --------------------
		% \phi_X
		% --------------------
		
		\draw[<->, thick] (9.5,1.8) -- (9.5,0.8);
		\node[annot, left] at (9.4,1.25) {$q\cdot \chi$};
		
		% -----------------
		
	
		
	
		% --------------------
		% FUNCIÓN TRABAJO METAL
		% --------------------
		\draw[<->, thick] (-2.4,0) -- (-2.4,1.165);
		\node[annot,left] at (-1.3,0.6) {$q \cdot \Phi_M$};
		
		
		% --------------------
		% FUNCIÓN TRABAJO GaAs
		% --------------------
		\draw[<->, thick] (8,0) -- (8,1.8);
		\node[annot,left] at (9,0.4) {$q\cdot \Phi_S$};
		
		
		% --------------------
		% EJES
		% --------------------
		\draw[axis] (-3,-3) -- (-3,2.4) node[above] {Energía};
		\draw[axis] (-3,-3) -- (10.4,-3) node[right] {Posición y};
		
		
		
		% --------------------
		% NIVEL DE VACÍO E0
		% --------------------
		\draw[thick] (-3,1.165) -- (3,1.165);
		\draw[band]
		(3,1.165)
		.. controls (3.2,1.165) and (4.4,1.8) ..
		(6,1.8)
		-- (10.3,1.8);
		\node[annot, right] at (10.4,1.8) {$E_0$};
		
		
			
		% --------------------
		% MATERIALES
		% --------------------
		\node[annot] at (-1,-3.6) {Metal: Titanio (Ti)};
		\node[annot] at (7,-3.6) {Semiconductor: GaAs tipo $N^+$};
		
	\end{tikzpicture}
	
\end{figure}


\subsection{Diagrama de bandas bajo régimen de corte}

La puerta está polarizada tan inversamente que el ancho del canal se hace nulo $d_{eff} = a - W_d(x) = 0 \,$ ya que se cumple $W_d(x) = a$, se define $V_P$ como la tensión $V_{GS}$ donde esto comienza a cumplirse. 

Se introduce la tensión $V_R =  V(x) - V_{GS} $ que es la responsable de aumentar la flexión en las bandas. Donde se define como referencia $V_S = 0 $ , por lo tanto $V(x) \in [0 , V_{DS}]$ y $V_{GS} = V_{G}$. A su vez al estar fuera del equilibrio el nivel de Fermi varia entre materiales, se introduce el concepto de cuasi-nivel de Fermi para cada nivel, $E_{f_{sc}}$ para el semiconductor y $E_{f_{m}}$ para el metal. El dispositivo no conducirá siempre y cuando las tensiones no sean excesivamente grandes, es decir no se supere el campo critico del semiconductor. 

\begin{figure}[H]
	\begin{tikzpicture}
		[
		>=Latex,
		axis/.style={->, thick},
		band/.style={thick},
		level/.style={dashed, thick},
		annot/.style={font=\small}
		]
		
		% --------------------
		% BANDA DE CONDUCCIÓN
		% --------------------
		\fill[gray!20]
		(3,2)
		.. controls (3.2,2) and (4.4,0.2) ..
		(7,0.2)
		-- (10,0.2)
		-- (10,-1)
		-- (7,-1)
		.. controls (4.4,-1) and (3.2,0.8) ..
		(3,0.8)
		-- cycle;
		
		\draw[band]
		(3,0.8)
		.. controls (3.2,0.8) and (4.4,-1) ..
		(7,-1)
		-- (10,-1);
		
		\node[annot] at (10.3,-1) {$E_C$};
		
		% --------------------
		% NIVEL DE FERMI SC
		% --------------------
		\draw[level] (3,-1.8) -- (10.3,-1.8);
		\node[annot, right] at (10.5,-1.8) {$E_{f_{sc}}$};
		
		% --------------------
		% NIVEL DE FERMI M
		% --------------------
		\draw[level] (-3,-0.6) -- (3,-0.6);
		\node[annot, right] at (-1.4,-0.3) {$E_{f_{m}}$};
		
		
		% --------------------
		% Altura barrera VR
		% --------------------
		\draw[<->, thick] (1.5,-1.8) -- (1.5,-0.6);
		\node[annot, right] at (0.5,-1.4) {$q \cdot V_R $};
		
		
		% --------------------
		% BANDA DE VALENCIA (MÁS ABAJO)
		% --------------------
		\fill[gray!40]
		(3,-4)
		-- (3,-2)
		.. controls (3.2,-2) and (4.4,-3.9) ..
		(7,-3.9)
		-- (10,-3.9)
		-- (10,-4)
		-- cycle;
		
		\draw[band]
		(3,-2)
		.. controls (3.2,-2) and (4.4,-3.9) ..
		(7,-3.9)
		-- (10,-3.9);
		
		\node[annot, right] at (10,-3.7) {$E_V$};
		
		% --------------------
		% INTERFAZ
		% --------------------
		\draw[thick] (3,-4.2) -- (3,2.4);
		\node[annot, above] at (3,2.4) {Interfaz};
		\node[annot, below] at (3,-4.2) {$y=0$};
		
		% --------------------
		% ALTURA DE BARRERA SCHOTTKY ΦB
		% --------------------
		\draw[level] (3,0.8) -- (7,0.8);
		\draw[<->, thick] (2.8,0.8) -- (2.8,-0.6);
		\node[annot, left] at (2.7,0) {$q\cdot \Phi_B$};
	
		% --------------------
		% \phi_X
		% --------------------
		
		\draw[<->, thick] (9.5,-1) -- (9.5,0.2);
		\node[annot, left] at (10.5,-0.5) {$q\cdot \chi$};
		
		% -----------------
		
		% --------------------
		% POTENCIAL INTERNO φ_bi
		% --------------------
		
		
		\draw[<->, thick] (7.5,-1) -- (7.5,0.8);
		\node[annot, right] at (7,0.9) {$q \cdot (\phi_{bi} + V_R )$};
		
		% --------------------
		% ANCHO DE AGOTAMIENTO W
		% --------------------
		
		\draw[level] (7,-4.2) -- (7,2.4);
		
		\draw[<->, thick] (3,-3.2) -- (7,-3.2);
		\node[annot, below] at (6,-3.2) {$W_d = a$};
		
			
		% --------------------
		% FUNCIÓN TRABAJO METAL
		% --------------------
		\draw[<->, thick] (-2.4,-0.6) -- (-2.4,2.0);
		\node[annot,left] at (-1.3,0.6) {$q \cdot \Phi_M$};
		
			
		% --------------------
		% FUNCIÓN TRABAJO GaAs
		% --------------------
		\draw[<->, thick] (8.5,-1.8) -- (8.5,0.2);
		\node[annot,left] at (9.5,-0.5) {$q\cdot \Phi_S$};
		
		
		% --------------------
		% EJES
		% --------------------
		\draw[axis] (-3,-4) -- (-3,2.4) node[above] {Energía};
		\draw[axis] (-3,-4) -- (10.4,-4) node[right] {Posición y};
		
		
		% --------------------
		% NIVEL DE VACÍO E0
		% --------------------
		\draw[thick] (-3,2) -- (3,2);
		\draw[band]
		(3,2)
		.. controls (3.2,2) and (4.4,0.2) ..
		(7,0.2)
		-- (10.3,0.2);
		\node[annot, right] at (10.4,0.2) {$E_0$};
		
		
		% --------------------
		% MATERIALES
		% --------------------
		\node[annot] at (-1,-4.6) {Metal: Titanio (Ti)};
		\node[annot] at (7,-4.6) {Semiconductor: GaAs tipo N};
		
	\end{tikzpicture}
	
	\caption{MESFET en modo corte: el canal queda completamente agotado.}
\end{figure}



\subsection{Diagrama de bandas bajo régimen de estrangulación}

	En este régimen se cumple $\, V_{GS} > V_P \quad \wedge\quad V_{DS} > V_{DS_{\text{SAT}}} = V_P + V_{GS} - V_{bi}$ , por lo tanto el canal es controlado por $V_{GS}$.

\begin{figure}[H]
	\centering
	
	% ---------- Imagen (izquierda) ----------
	\begin{minipage}[c]{0.45\textwidth}
		\centering
		\includegraphics[width=\linewidth]{Img/diagrama_bandas_alolargodelcanal}
		\captionof{figure}{Diagrama de bandas en $W_D(0)=W_{Ds}$ y $W_D(L)=W_{Dd}$ \parencite{mishra_semiconductor_2008}.}
		\label{fig:diagramabandasalolargodelcanal}
	\end{minipage}
	\hfill
	% ---------- Texto (derecha) ----------
	\begin{minipage}[c]{0.5\textwidth}
	
		Se observa en la figura \ref{fig:diagramabandasalolargodelcanal} que el ancho de la zona de vaciamiento cumple
		$W_{d0} < W_d(0) < a$, mientras que $W_d(L) = a$.
		A medida que $V_{DS}$ aumenta más allá de $V_{DS_{\text{SAT}}}$,
		la longitud de la región de empobrecimiento crece a lo largo del canal, pero el nivel de $I_D$ permanece constante, razón por la cual $V_{DS}$ idealmente no incide en $I_D$ en esta región.
		
		Los portadores atraviesan la región de empobrecimiento debido
		al arrastre generado por el campo eléctrico longitudinal.
	\end{minipage}
	
\end{figure}


\begin{figure}[H]
		\begin{tikzpicture}
		[
		>=Latex,
		axis/.style={->, thick},
		band/.style={thick},
		level/.style={dashed, thick},
		annot/.style={font=\small}
		]
		
		% --------------------
		% BANDA DE CONDUCCIÓN
		% --------------------
		\fill[gray!20]
		(3,2.0)
		.. controls (3.2,2.0) and (4.4,1.2) ..
		(6.5,1.2)
		-- (10,1.2)
		-- (10,-0.6)
		-- (6.5,-0.6)
		.. controls (4.4,-0.6) and (3.2,0.6) ..
		(3,0.6)
		-- cycle;
		
		\draw[band]
		(3,0.6)
		.. controls (3.2,0.6) and (4.4,-0.6) ..
		(6.5,-0.6)
		-- (10,-0.6);
		
		\node[annot] at (10.3,-0.6) {$E_C$};
		
			% --------------------
		% NIVEL DE FERMI SC
		% --------------------
		\draw[level] (3,-1.5) -- (10.3,-1.5);
		\node[annot, right] at (10.5,-1.5) {$E_{f_{sc}}$};
		
		% --------------------
		% NIVEL DE FERMI M
		% --------------------
		\draw[level] (-3,-0.8) -- (3,-0.8);
		\node[annot, right] at (-1.4,-0.6) {$E_{f_{m}}$};
		
		
		% --------------------
		% Altura barrera VR
		% --------------------
		\draw[<->, thick] (1.5,-0.8) -- (1.5,-1.5);
		\node[annot, right] at (1.6,-1.4) {$q \cdot V_R$};
	
		% --------------------
		% BANDA DE VALENCIA (MÁS ABAJO)
		% --------------------
		\fill[gray!40]
		(3,-4)
		-- (3,-1.6)
		.. controls (3.2,-1.6) and (4.4,-2.9) ..
		(6.5,-2.9)
		-- (10,-2.9)
		-- (10,-4)
		-- cycle;
		
		\draw[band]
		(3,-1.6)
		.. controls (3.2,-1.6) and (4.4,-2.9) ..
		(6.5,-2.9)
		-- (10,-2.9);
		
		\node[annot, right] at (10.3,-2.9) {$E_V$};
		
		% --------------------
		% INTERFAZ
		% --------------------
		\draw[thick] (3,-4.2) -- (3,2.4);
		\node[annot, above] at (3,2.4) {Interfaz};
		\node[annot, below] at (3,-4.2) {$y=0$};
		
		% --------------------
		% ALTURA DE BARRERA SCHOTTKY ΦB
		% --------------------
		\draw[level] (3,0.6) -- (6.5,0.6);
		\draw[<->, thick] (2.8,0.6) -- (2.8,-0.8);
		\node[annot, left] at (2.7,0) {$q\cdot \Phi_B$};
		
		
		% --------------------
		% \phi_X
		% --------------------
		
		\draw[<->, thick] (9.5,-0.6) -- (9.5,1.2);
		\node[annot, left] at (10.35,0.5) {$q\cdot \chi$};
		
		% -----------------
		
		% --------------------
		% POTENCIAL INTERNO φ_bi
		% --------------------
		
		
		\draw[<->, thick] (6.5,-0.6) -- (6.5,0.6);
		\node[annot, right] at (6.6,0.3) {$q \cdot (\phi_{bi} + V_R )$};
		
		% --------------------
		% ANCHO DE AGOTAMIENTO W
		% --------------------
		
		\draw[level] (6.5,-4.2) -- (6.5,2.4);
		
		\draw[<->, thick] (3,-3.2) -- (6.5,-3.2);
		\node[annot, below] at (4.5,-3.2) {$W_d$};
		
		% --------------------
		% FUNCIÓN TRABAJO METAL
		% --------------------
		\draw[<->, thick] (-2.4,-0.8) -- (-2.4,2.0);
		\node[annot,left] at (-1.3,0.6) {$q \cdot \Phi_M$};
		
		
		% --------------------
		% FUNCIÓN TRABAJO GaAs
		% --------------------
		\draw[<->, thick] (8,-1.5) -- (8,1.2);
		\node[annot,left] at (9,-1) {$q\cdot \Phi_S$};
		
		
		% --------------------
		% EJES
		% --------------------
		\draw[axis] (-3,-4) -- (-3,2.4) node[above] {Energía};
		\draw[axis] (-3,-4) -- (10.4,-4) node[right] {Posición y};
		
		
		% --------------------
		% NIVEL DE VACÍO E0
		% --------------------
		\draw[thick] (-3,2.0) -- (3,2.0);
		\draw[band]
		(3,2.0)
		.. controls (3.2,2.0) and (4.4,1.2) ..
		(6,1.2)
		-- (10.3,1.2);
		\node[annot, right] at (10.4,1.2) {$E_0$};
		
		
		% --------------------
		% MATERIALES
		% --------------------
		\node[annot] at (-1,-4.6) {Metal: Titanio (Ti)};
		\node[annot] at (7,-4.6) {Semiconductor: GaAs tipo N};
		
	\end{tikzpicture}
	\caption{MESFET en modo estrangulación: inicio de saturación del canal.}
\end{figure}


\subsection{Diagrama de bandas bajo régimen óhmico}

Cuando ocurre $0 < V_{DS} < V_{DS_{[SAT]}} $ y $V_{GS} > V_P$, en este escenario el ancho de vaciamiento cumple $ 0 < W_d(x) < W_{d0} < a$ para todo el canal.

\begin{figure}[H]
	\begin{tikzpicture}
	[
	>=Latex,
	axis/.style={->, thick},
	band/.style={thick},
	level/.style={dashed, thick},
	annot/.style={font=\small}
	]
	
	% --------------------
	% BANDA DE CONDUCCIÓN
	% --------------------
	\fill[gray!20]
	(3,2)
	.. controls (3.2,2) and (4.4,1.7) ..
	(6.5,1.7)
	-- (10,1.7)
	-- (10,0.3)
	-- (6.5,0.3)
	.. controls (4.4,0.3) and (3.2,0.6) ..
	(3,0.6)
	-- cycle;
	
	\draw[band]
	(3,0.6)
	.. controls (3.2,0.6) and (4.4,0.3) ..
	(6.5,0.3)
	-- (10,0.3);
	
	\node[annot] at (10.3,0.3) {$E_C$};
	
	% --------------------
	% NIVEL DE FERMI SC
	% --------------------
	\draw[level] (3,-0.3) -- (10.3,-0.3);
	\node[annot, right] at (10.5,-0.3) {$E_{f_{sc}}$};
	
	% --------------------
	% NIVEL DE FERMI M
	% --------------------
	\draw[level] (-3,-0.8) -- (3,-0.8);
	\node[annot, right] at (-1.4,-0.6) {$E_{f_{m}}$};
	
	

	% --------------------
	% BANDA DE VALENCIA (MÁS ABAJO)
	% --------------------
	\fill[gray!40]
	(3,-4)
	-- (3,-1.6)
	.. controls (3.2,-1.6) and (4.4,-1.9) ..
	(6.5,-1.9)
	-- (10,-1.9)
	-- (10,-4)
	-- cycle;
	
	\draw[band]
	(3,-1.6)
	.. controls (3.2,-1.6) and (4.4,-1.9) ..
	(6.5,-1.9)
	-- (10,-1.9);
	
	\node[annot, right] at (10.3,-2.9) {$E_V$};
	
	% --------------------
	% INTERFAZ
	% --------------------
	\draw[thick] (3,-4.2) -- (3,2.4);
	\node[annot, above] at (3,2.4) {Interfaz};
	\node[annot, below] at (3,-4.2) {$y=0$};
	
	% --------------------
	% ALTURA DE BARRERA SCHOTTKY ΦB
	% --------------------

	\draw[<->, thick] (2.8,0.6) -- (2.8,-0.8);
	\node[annot, left] at (2.7,0) {$q\cdot \Phi_B$};
	
	
	% --------------------
	% \phi_X
	% --------------------
	
	\draw[<->, thick] (9.5,0.3) -- (9.5,1.7);
	\node[annot, left] at (10.35,1) {$q\cdot \chi$};
	
	% -----------------
	
	% --------------------
	% POTENCIAL INTERNO φ_bi
	% --------------------
	
	

	% --------------------
	% ANCHO DE AGOTAMIENTO W
	% --------------------
	
	\draw[level] (6.5,-4.2) -- (6.5,2.4);
	
	\draw[<->, thick] (3,-3.2) -- (6.5,-3.2);
	\node[annot, below] at (4.5,-3.2) {$W_d$};
	
	% --------------------
	% FUNCIÓN TRABAJO METAL
	% --------------------
	\draw[<->, thick] (-2.4,-0.8) -- (-2.4,2.0);
	\node[annot,left] at (-1.3,0.6) {$q \cdot \Phi_M$};
	
	
	% --------------------
	% FUNCIÓN TRABAJO GaAs
	% --------------------
	\draw[<->, thick] (8,-0.3) -- (8,1.7);
	\node[annot,left] at (9,1) {$q\cdot \Phi_S$};
	
	
	% --------------------
	% EJES
	% --------------------
	\draw[axis] (-3,-4) -- (-3,2.4) node[above] {Energía};
	\draw[axis] (-3,-4) -- (10.4,-4) node[right] {Posición y};
	
	
	% --------------------
	% NIVEL DE VACÍO E0
	% --------------------
	\draw[thick] (-3,2.0) -- (3,2.0);
	\draw[band]
	(3,2.0)
	.. controls (3.2,2.0) and (4.4,1.7) ..
	(6,1.7)
	-- (10.3,1.7);
	\node[annot, right] at (10.4,1.7) {$E_0$};
	
	
	% --------------------
	% MATERIALES
	% --------------------
	\node[annot] at (-1,-4.6) {Metal: Titanio (Ti)};
	\node[annot] at (7,-4.6) {Semiconductor: GaAs tipo N};
	
\end{tikzpicture}
	\caption{MESFET en modo óhmico: conducción lineal del canal.}
\end{figure}



\subsection{Flexión dentro del Canal}

A lo largo del canal, la aplicación de un potencial $V_{DS}$ introduce un potencial longitudinal V(x) que se manifiesta como una inclinación cuasi-lineal de las bandas del semiconductor, como se observa en la figura \ref{fig:inclinacionbandasdebidoavds}. Este efecto no está asociado a la juntura Schottky, sino al campo eléctrico longitudinal responsable del transporte de portadores entre source y drain.

\begin{figure}[H]
	\centering
	\includegraphics[width=0.7\linewidth]{Img/inclinación_bandas_debido_aVDS}
	\caption{Diagrama de bandas a lo largo del canal \parencite{mishra_semiconductor_2008}.}
	\label{fig:inclinacionbandasdebidoavds}
\end{figure}



\section{Obtención del modelo completo}

Utilizando el sistema de coordenadas y notación inspirados en la figura \ref{fig:simbolomasesquema}, tomando la referencia espacial $y = 0$ en la unión metal-semiconductor y asumiendo ciertas hipótesis, se puede hallar la expresión de la corriente partiendo :

\texttt{Hipotesis [1] : ${N_D}^+ = N_D $ (impurezas totalmente ionizadas) y semiconductor no degenerado.}

\texttt{Hipotesis [2] : Solo hay carga en la región de vaciamiento.}


 Los electrones libres del semiconductor en las inmediaciones de la unión se transfieren hacia la superficie del metal, formando una carga superficial negativa. En el semiconductor se establece una zona de átomos donadores ionizados que conforman una carga positiva. Este proceso al alcanzar el equilibrio termodinámico (ETD) permite determinar, en reposo, la densidad de carga, el campo electrostático que se forma y el potencial eléctrico.

\subsection{Condición de neutralidad}

Considerando el semiconductor tipo n dopado uniformemente en ETD, se supone que la concentración de aceptadores es cero $N_a = 0$ y que el total de los átomos de Si donadores libero su electrón $ {N_d}^{+} = N_d$ y $p_0 \ll N_d$.

\begin{equation}
	\rho(x) = q\left( {N_d}^{+} + p_0 - N_a^{-} - n_0 \right)
	\simeq q\left( N_d - n_0 \right) = 0
\end{equation}

\subsection{Densidad de carga}

La carga se formara en el semiconductor y en la superficie del metal, formando una región superficial con carga negativa y una región en el semiconductor de carga positiva de acuerdo a (\ref{equation:def_densidad_carga}) utilizando las hipótesis [1] y [2].

\begin{figure}[H]
	\centering
	
	% --------- Expresión analítica (izquierda) ----------
	\begin{minipage}[c]{0.45\textwidth}
		\centering
		\begin{equation}\label{equation:def_densidad_carga}
			\rho(x,y) =
			\begin{cases}
				0 & y < 0 \\[4pt]
				-q N_d W(x) & y = 0 \\[4pt]
				q N_d W(x) & 0 < y \le W(x) \\[4pt]
				0 & y > W(x)
			\end{cases}
		\end{equation}
	\end{minipage}
	\hfill
	% --------- Gráfico (derecha) ----------
	\begin{minipage}[c]{0.5\textwidth}
		\centering
		\begin{tikzpicture}[
			>=Latex,
			axis/.style={->, thick},
			charge/.style={fill=gray!40, draw=black},
			annot/.style={font=\small}
			]
			
			% Ejes
			\draw[axis] (-3,0) -- (3,0) node[right] {$y$};
			\draw[axis] (0,-3) -- (0,3) node[above] {$\rho(y)$};
			
			% Región positiva
			\draw[charge] (0,0) rectangle (2.5,2);
			\node[annot] at (1.25,1) {$qN_d W(x)$};
			
			% Región negativa (carga superficial)
			\draw[charge] (-0.1,-3) rectangle (0,0);
			\node[annot,left] at (-0.15,-1.3) {$-qN_d W(x)$};
			
			
			% Etiquetas de materiales
			\node[annot, anchor=east] at (-0.8,2.5) {Metal: Ti};
			\node[annot, anchor=west] at (0.8,2.5) {SC: GaAs tipo N};
			
			
			% W(x)
			\draw[<->] (0,-0.2) -- (2.5,-0.2)
			node[midway,below] {$W(x)$};
			
		\end{tikzpicture}
	\end{minipage}
	
	\caption{Expresión analítica y representación gráfica de la densidad de carga $\rho(x,y)$.}
\end{figure}



\subsection{Campo eléctrico}

Para obtener el campo eléctrico (\ref{equation:campo_electrico}), utilizamos la ley de Gauss (\ref{equation:ley_gauss}):

\begin{equation}\label{equation:ley_gauss}
	\mathcal{E} = \frac{Q}{\epsilon_s}
\end{equation} 

\begin{figure}[h]
	\centering
	
	% --------- Expresión analítica (izquierda) ----------
	\begin{minipage}[c]{0.45\textwidth}
		\centering
		\begin{equation}\label{equation:campo_electrico}
			\mathcal{E}(x,y) =
			\begin{cases}
				0 & y \le 0 \\[4pt]
				\dfrac{q N_D}{\epsilon_s}\,[y - W(x)] 
				& 0 < y \le W(x) \\[6pt]
				0 & y > W(x)
			\end{cases}
		\end{equation}
	\end{minipage}
	\hfill
	% --------- Gráfico (derecha) ----------
	\begin{minipage}[c]{0.5\textwidth}
		\centering
		\begin{tikzpicture}[
			>=Latex,
			axis/.style={->, thick},
			annot/.style={font=\small}
			]
			
			% Ejes
			\draw[axis] (-3,0) -- (3,0) node[right] {$y$};
			\draw[axis] (0,-3) -- (0,2) node[above] {$\mathcal{E}(y)$};
			
			% Campo eléctrico en la región de vaciamiento (lineal)
			\draw[thick] (0,-2) -- (2.5,0);
			
			% Línea de referencia en y = W(x)
		%	\draw[dashed] (2.5,-2) -- (2.5,0);
			\node[annot, below] at (2.5,0.5) {$W(x)$};
			
			% Etiquetas de materiales (posiciones solicitadas)
			\node[annot, anchor=east] at (-0.8,1.2) {Metal: Ti};
				\node[annot, anchor=east] at (-0.1,-2) {$\mathcal{E}_{MAX}$};
			\node[annot, anchor=west] at (0.8,1.2) {SC: GaAs tipo N};
			
			% Pendiente
			\node[annot] at (1.8,-1.2)
			{$\displaystyle \frac{qN_D}{\epsilon_s}$};
			
		\end{tikzpicture}
	\end{minipage}
	
	\caption{Expresión analítica y representación gráfica del campo eléctrico $\mathcal{E}(x,y)$ en la unión metal--GaAs.}
\end{figure}

\subsection{Potencial eléctrico}

Dada la relación entre el campo eléctrico y el gradiente del potencial eléctrico (\ref{equation:relacion_P_E}) se obtiene el potencial eléctrico, al integrar el potencial se debe tener en consideración una referencia inicial conveniente como $\phi (y = 0 ) = 0$ y inicialmente teniendo solo en consideración el potencial de juntura $\phi [y = W(x) ] = \phi_{bi}$, analizado en mas profundidad en la sección de bandas de energía del informe. 

\begin{equation}\label{equation:relacion_P_E}
	\mathcal{E} = - \nabla \phi
\end{equation}

\begin{figure}[H]
	\centering
	
	% --------- Expresión analítica (izquierda) ----------
	\begin{minipage}[c]{0.45\textwidth}
		\centering
		\begin{equation}\label{equation:potencial_electrostatico}
			\phi(x,y) =
			\begin{cases}
				0 & y \le 0 \\[4pt]
				-\dfrac{q N_D}{2\epsilon_s}\,[y - W(x)]^2 + \phi_{bi}
				& 0 < y \le W(x) \\[6pt]
				\phi_{bi} & y > W(x)
			\end{cases}
		\end{equation}
	\end{minipage}
	\hfill
	% --------- Gráfico (derecha) ----------
	\begin{minipage}[c]{0.5\textwidth}
		\centering
		\begin{tikzpicture}[
			>=Latex,
			axis/.style={->, thick},
			annot/.style={font=\small},
			scale=1
			]
			
			% Ejes
			\draw[axis] (-3,0) -- (4,0) node[right] {$y$};
			\draw[axis] (0,-0.5) -- (0,2.5) node[above] {$\phi(y)$};
			
			% Potencial en la región de vaciamiento (parábola)
			\draw[thick, domain=0:3.5, samples=100]
			plot (\x, -{0.08*(\x-3.5)^2 + 1});
			
			% Línea constante para y > W(x)
			\draw[thick] (3,1) -- (3,1);
			\node[annot, right] at (3.3,1.3) {$\phi_{bi}$};
			
			% Línea vertical en W(x)
			\draw[dashed] (3,0) -- (3,1);
			\node[annot, below] at (3,0) {$W(x)$};
			
			% Etiquetas de materiales (posiciones solicitadas)
			\node[annot, anchor=east] at (-0.8,2) {Metal: Ti};
			\node[annot, anchor=west] at (0.8,2) {SC: GaAs tipo N};
			
			% Valor en y = 0
			\node[annot, below left] at (0,0) {$0$};
			
		\end{tikzpicture}
	\end{minipage}
	
	\caption{Expresión analítica y representación gráfica del potencial electrostático $\phi(x,y)$ en la unión metal--GaAs.}
\end{figure}

\subsection{Ancho de zona de vaciamiento}

Relacionando el campo eléctrico (\ref{equation:campo_electrico}) junto con el potencial (\ref{equation:potencial_electrostatico}) se puede despejar el ancho de la zona de agotamiento o vaciamiento en reposo y equilibrio térmico.

\begin{equation}\label{equation: W_d_reposo}
	W_{d0} = \sqrt{\frac{2 \cdot \epsilon_s \cdot \phi_{bi}}{q\cdot N_D}}
\end{equation} 


Al considerar la polarización del dispositivo se produce un potencial longitudinal $V(x) \in [0 ,V_{DS}]$ y el potencial externo trasversal al canal entre \textit{gate} y \textit{source} $V_{GS}$, que afectaran a la modulación de la zona de vaciamiento, nombramos $ \phi_{bi} = V_{bi}$.

\begin{equation}\label{equation: W_d_polarizado}
	W_d(x) = \sqrt{\frac{2 \cdot \epsilon_s \cdot [  V_{bi} - V_{GS} + V(x)] }{q\cdot N_D}}
\end{equation} 

\subsection{Obtención de $V_{P}$}

Partiendo de (\ref{equation: W_d_reposo}) reemplazando $W_d(x) = a $ se obtiene la tensión para la cual el dispositivo comienza a conducir: 

\begin{equation}\label{equation:obtenicon_VP}
	W_d = \sqrt{\frac{2 \cdot \epsilon_s \cdot V_P }{q\cdot N_D}} = a 
\end{equation} 

Llegando finalmente a (\ref{equation:obtenicon_VP_2})
\begin{equation}\label{equation:obtenicon_VP_2}
		 V_P = \frac{a^2 \cdot q\cdot N_D}{2 \cdot \epsilon_s }
\end{equation}



\subsection{Obtención de $V_{DS_{[sat]}}$}
Mientras que evaluando la expresión del ancho de vaciamiento (\ref{equation: W_d_polarizado}) en la región del terminal \textit{drain} se obtiene la tensión de saturación, ya que se debe cumplir $V(x = L ) = V_{DS} \quad \wedge \quad W_d(x = L ) = a$:

\begin{equation}\label{equation:obtension_VDS_SAT}
	W_d(x = L ) = \sqrt{\frac{2 \cdot \epsilon_s \cdot [  V_{bi} - V_{GS} + V_{DS }] }{q\cdot N_D}} = a 
\end{equation} 

\begin{equation}\label{equation:obtension_VDS_SAT_2}
	 [  V_{bi} - V_{GS} + V_{DS }]  = \frac{a^2 \cdot q\cdot N_D}{2 \cdot \epsilon_s}
\end{equation} 

\begin{equation}\label{equation:obtension_VDS_SAT_3}
	 V_{DS}  = \frac{a^2 \cdot q\cdot N_D}{2 \cdot \epsilon_s} + V_{GS} -  V_{bi}
\end{equation} 

Se puede apreciar que se forma la tensión de umbral $V_P = \frac{a^2 \cdot q\cdot N_D}{2 \cdot \epsilon_s}$, llegando finalmente a la expresión (\ref{equation:obtension_VDS_SAT_4}):

\begin{equation}\label{equation:obtension_VDS_SAT_4}
	V_{DS_{[sat]}} = V_P + V_{GS} -  V_{bi}
\end{equation} 


\section{Corriente}

El dispositivo bajo conducción forma una densidad de corriente de arrastre de portadores mayoritarios $	J_{arr, n} $ en el eje \textit{x} , debido al campo aplicado $\xi$ y asumiendo $ n \approx N_D$, de acuerdo a la expresión (\ref{equation:J_corriente_deduccion_1}) :

\begin{equation}\label{equation:J_corriente_deduccion_1}
	J_{arr, n} = q \cdot n \cdot \mu_n \cdot \xi  =  q \cdot N_D \cdot \mu_n \cdot (- \frac{dV}{dx})
\end{equation}

Se puede utilizar nuevamente la relación (\ref{equation:relacion_P_E}) y utilizar el gradiente del potencial.

De la densidad de corriente, se puede despejar la corriente que atraviesa el canal, la cual denominaremos como $I_D$, si conocemos su área $A_{ch} = Z \cdot d_eff = Z\cdot [a - W(x)]$.

\begin{equation}
	I_D =A_{ch}\cdot J_{arr, n} = (Z\cdot [a - W(x)]) \cdot J_{arr, n} 
\end{equation}


\begin{equation}
	I_D = (Z\cdot [a - W(x)]) \cdot q \cdot N_D \cdot \mu_n \cdot (- \frac{dV}{dx})
\end{equation}

Debido que la corriente se define positiva entrante al dispositivo.

\begin{equation}
		I_D =  Z \cdot q \cdot N_D \cdot \mu_n \cdot[a - W(x)] \cdot ( \frac{dV}{dx})
\end{equation}


\begin{equation}
	I_D \cdot dx = Z\cdot q \cdot N_D \cdot \mu_n \cdot a \cdot dV - Z\cdot q \cdot N_D \cdot \mu_n \cdot W(x) \cdot dV  
\end{equation}

\begin{equation}
	\int_0^L I_D \cdot dx =  Z\cdot q \cdot N_D \cdot \mu_n \cdot a 	\int_0^{V_{DS}}  \; dV - Z\cdot q \cdot N_D \cdot \mu_n \cdot 	\int_0^{V_{DS}}  W(x) \cdot dV 
\end{equation}



\begin{equation}
	 I_D \cdot L = Z\cdot q \cdot N_D \cdot \mu_n \cdot 	\left[ 	\int_0^{V_{DS}}  \; dV -  \frac{1}{a} \int_0^{V_{DS}}  W(x) \cdot dV\right]   
\end{equation}

Resolviendo la integral:

\begin{equation}
 \frac{1}{a}	\int_0^{V_{DS}}  W(x) \cdot dV =  \frac{1}{a}	\int_0^{V_{DS}} \sqrt{\frac{2 \cdot \epsilon_s \cdot [  V_{bi} - V_{GS} + V(x)] }{q\cdot N_D}}\cdot dV
\end{equation}

Se puede despejar $V_P$ de la siguiente forma: 
\begin{equation}
  \sqrt{ \frac{2 \cdot \epsilon_s}{q\cdot N_D \cdot a^2}}	\int_0^{V_{DS}} \sqrt{ (V_{bi} - V_{GS} + V(x))} \; dV
\end{equation}

\begin{equation}
	\sqrt{ \frac{1}{V_P}} \cdot \int_0^{V_{DS}} \sqrt{ (V_{bi} - V_{GS} + V(x))} \; dV
\end{equation}

\begin{equation}
	 \frac{1}{\sqrt{V_P}} \cdot \frac{2}{3} \left[ (V_{bi} - V_{GS} + V_{DS} )^{\frac{3}{2}} - ( V_{bi} - V_{GS})^{\frac{3}{2}} \right]
\end{equation}


Definiendo $g_o = \frac{Z\cdot q \cdot N_D \cdot \mu_n \cdot a}{L}$ como la conductancia de canal completo, se puede acomodar la expresión como:


\begin{equation}
	I_D = g_o \cdot	\left\{ V_{DS}  - \frac{2}{3 \cdot \sqrt{V_P}} \cdot \left[ (V_{bi} - V_{GS} + V_{DS} )^{\frac{3}{2}} - ( V_{bi} - V_{GS})^{\frac{3}{2}} \right]\right\} 
\end{equation}


\begin{tcolorbox}[colback=black!5!white, colframe=black!75!black, title=Expresión de la corriente ID completa en función de los parámetros de control y del circuito externo]


		\begin{equation}
				I_D = g_o \cdot	\left\{ V_{DS}  - \frac{2}{3 \cdot \sqrt{V_P}} \cdot \left[ (V_{bi} - V_{GS} + V_{DS} )^{\frac{3}{2}} - ( V_{bi} - V_{GS})^{\frac{3}{2}} \right] \right\} 
		\end{equation}
	

\end{tcolorbox}



\section{Comparación}

\section{Simulación de curvas}

\subsection{Curva de transferencia}


\begin{figure}[H]
	\centering
	\includegraphics[width=\linewidth]{Img/curva_transferencia}
	\caption{}
	\label{fig:curvatransferencia}
\end{figure}


\subsection{Curva de salida}

\begin{figure}[H]
	\centering
	\includegraphics[width=\linewidth]{Img/curva_de_salida}
	\caption{}
	\label{fig:curvadesalida}
\end{figure}


\section{Efectos no ideales}

En el análisis previo se contemplo un comportamiento ideal, sin contemplar corrientes de fuga, parámetros uniformes, una longitud de canal constante y una movilidad constante.

Cuando un MESFET está polarizado en la región de estrangulamiento, la longitud efectiva del canal eléctrico se determina por la tensión $V_{DS}$. Este efecto no ideal se denomina modulación de longitud de canal. Además, cuando un transistor está polarizado cerca o en la región de estrangulamiento, el campo eléctrico en el canal puede alcanzar la magnitud suficiente para que los portadores mayoritarios alcancen su velocidad de saturación. En este punto, la movilidad deja de ser constante. La magnitud de la corriente de compuerta afectará la impedancia de entrada, lo cual puede ser necesario tener en cuenta en el diseño del circuito \parencite{neamen2012}.

\subsection{Modulación del largo del canal}

Para los MESFET de alta frecuencia, las longitudes de canal típicas son del orden de 1 $\mu$m, la modulación de la longitud del canal adquieren gran importancia en dispositivos de canal corto 
\parencite{neamen2012}.

\subsection{Saturación de la velocidad de arrastre}
\subsection{Región subumbral}
\subsection{Efectos de corriente de gate}

\section{Conclusión} 


Ejemplo referenicias\\
El comportamiento del MOSFET en saturación está bien modelado
por la ecuación cuadrática \parencite{sedra2015}.

Según \textcite{boylestad2013}, la tensión umbral depende del dopado.
El comportamiento de los portadores en semiconductores puede
modelarse a partir de principios cuánticos \parencite{neamen2012}.

Según \textcite{sze2021}, el MOSFET moderno requiere modelos
avanzados de canal corto.

El nivel de Fermi se introduce naturalmente desde la física
del estado sólido \parencite{mckelvey1996}.

\printbibliography

\section{Apéndice}

\subsection{Código}

\end{document}