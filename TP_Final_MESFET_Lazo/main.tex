\documentclass[a4paper]{article}
\usepackage[margin=1.5cm,top=1.5cm,bottom=1.5cm,a4paper]{geometry}
\usepackage{graphicx} % Required for inserting images
\usepackage{titlesec} % Required for customizing section titles
\usepackage[margin=1.5cm,top=1.5cm, bottom=1.5cm]{geometry} % Ajusta los márgenes aquí
\usepackage{multicol} % Required for multicols environment
\usepackage{parskip}
\usepackage{etoolbox}
\usepackage{float}
\usepackage{amsmath}
\usepackage{dingbat}
\usepackage{booktabs} 
\usepackage{caption}
\usepackage[utf8]{inputenc}
\usepackage[T1]{fontenc}
\usepackage{amssymb}
\usepackage[spanish]{babel}

\usepackage{tikz}
\usetikzlibrary{arrows.meta, positioning}
\usetikzlibrary{babel}

\usepackage{csquotes}

\usepackage[
  backend=biber,
  style=apa,
  citestyle=apa
]{biblatex}

\addbibresource{referencias.bib}



%\captionsetup[table]{name = Tabla}
 
% Redefinir el comando \thesection para usar números romanos
%\renewcommand{\thesection}{\Roman{section}}
% Redefinir el comando \thesubsection para usar números romanos
%\renewcommand{\thesubsection}{\thesection.\Roman{subsection}}


\renewcommand{\thesection}{\arabic{section}}
\renewcommand{\thesubsection}{\thesection.\arabic{subsection}}
\renewcommand{\thesubsubsection}{\thesubsection.\arabic{subsubsection}}


% Personalización de la sección para que comience desde la izquierda
\titleformat{\section}[hang]{\normalfont\Large\bfseries}{\thesection}{1em}{}

\begin{document}
\setcounter{page}{1}
\begin{minipage}[h]{1\textwidth}

\title{\includegraphics[height = 1.5cm]{logofiuba.png} % Logo en la izquierda del encabezado  
\\(TB070) Dispositivos Semiconductores \\Trabajo Práctico Final : Transistor MESFET }
\author{106213 Sebastián Lazo  ( slazo@fi.uba.ar ) }
\date{26 de Febrero 2026}

\maketitle

 \end{minipage}

\section{Resumen}

En el siguiente informe se utilizan los conocimientos adquiridos a lo largo de la materia Dispositivos Semiconductores con el objetivo de realizar una investigación y análisis en torno al dispositivo conocido como MESFET ( Metal-Semiconductor Field-Effect Transistor ). Se aborda su estructura, parámetros característicos y aplicaciones típicas, modos de operación. Ademas, se presentan diagramas de bandas para cada modo de operación, curvas características y se discute los efectos no ideales que influyen en su comportamiento real.

\section{Descripción del Dispositivo}

El transistor MESFET es un dispositivo multijuntura de tres terminales: fuente, drenaje y compuerta, cuya estructura forma un canal por el cual fluyen los portadores desde el terminal fuente hacia el terminal de drenaje. La conductividad del canal es modulada por el potencial eléctrico aplicado en el terminal de compuerta. Puede clasificarse como una variación del transistor MOSFET, con la diferencia de que no posee una capa de óxido entre la compuerta y el sustrato, característica que también comparte con el transistor JFET. Con respecto a este último, difieren en el material de la compuerta, ya que en el MESFET no es de material semiconductor, sino metálica, empleándose metales como aluminio, titanio, oro, níquel o platino.

Los transistores MOSFET pueden no ser apropiados si se desea fabricar un transistor utilizando ciertos materiales semiconductores, como en el caso del arseniuro de galio (GaAs), dado que en la interfase entre el GaAs y un aislante como el oxido se forman un gran número de trampas superficiales que inhiben la acción compuerta. Además, las compuertas de unión no se difunden con tanta facilidad dentro del GaAs debido a la inestabilidad del material semiconductor a altas temperaturas, por lo tanto no se realizan uniones P-N con facilidad. Sin embargo, dado que el GaAs tiene una mayor movilidad que el silicio, su empleo es aconsejable en aplicaciones que requieren rápidas velocidades de conmutación. Por consiguiente, se utilizan estructuras del tipo MESFET para cubrir esas necesidades \parencite{StreetmanBanerjee_SolidStateDevices_2006}. Otros semiconductores utilizados para fabricar MESFET's son nitruro de galio (GaN), carburo de silicio de estructura cristalina (4H-SiC) y fosforo de indio (InP), las caracteristicas de los mismos se encuentran en la tabla de la figura \ref{fig:tablapropsctransistores}.


\begin{figure}[H]
	\centering
	\includegraphics[width=0.7\linewidth]{"../../../../Imágenes/Capturas de pantalla/TablaProp_SC_transistores"}
	\caption{Tabla con las caracteristicas princiáles de los semiconductores y el tipo de transistor fabricado con ellos \parencite{giannini2009high}.}
	\label{fig:tablapropsctransistores}
\end{figure}

El MESFET fue propuesto y demostrado por primera vez por Mead en 1966. Poco después, Hooper y Lehrer informaron sobre su rendimiento en dispositivos de microondas en 1967, utilizando una capa epitaxial de GaAs sobre un sustrato de GaAs semiaislante \parencite{sze2021}.



\subsection{Estructura}

Los MESFET's se construyen a partir de una delgada capa epitaxial de GaAs de tipo N dopada con impurezas donadoras como pueden llegar a ser Silicio, Azufre o Selenio depositada sobre un sustrato semi-aislante tambien de GaAs pero dopado intencionalmente con Cromo, que se comporta como un único aceptor cerca del centro de la banda prohibida de energía, con el objetivo de obtener un semi-aislante con una resistividad de ordenes de hasta hasta $10^9 \frac{\Omega}{cm}$ \parencite{neamen2012}. Sobre esta capa se definen las tres terminales del dispositivo: \textit{source}, \textit{drain} y \textit{gate} (fuente, drenaje y compuerta), como se aprecia en la Figura \ref{fig:estr_Belove_MESFET}.



\begin{figure}[H]
    \centering

    \begin{minipage}{0.48\textwidth}
        \centering
        \includegraphics[width=\linewidth]{Img/estructura_MESFET.png}
        \caption{Estructura del transistor MESFET de GaAs \parencite{StreetmanBanerjee_SolidStateDevices_2006}.}
        \label{fig:estr_Belove_MESFET}
    \end{minipage}
    \hfill
    \begin{minipage}{0.48\textwidth}
        \centering
        \includegraphics[width=\linewidth]{Img/estructura_MESFET_SZE.png}
        \caption{Estructura del transistor MESFET, se observa la apertura neta del canal $b$ controlada por el ancho de agotamiento $W_D$ \parencite{sze2021}.}
        \label{fig:estr_Sze_MESFET}
    \end{minipage}

\end{figure}

Sobre el sustrato semi-aislante se encuentra la región activa del dispositivo, conformada por la capa de GaAs dopada ligeramente, en ella se forma el canal por el cual circulan los portadores mayoritarios cuando el dispositivo está en conducción, esto se aprecia en la figura \ref{fig:neamen_strucutre}.

En las zonas correspondientes a \textit{source} y \textit{drain}, esta misma capa se dopa fuertemente con las mismas impurezas de tipo N para obtener contactos óhmicos de baja resistencia, facilitando así la inyección y recolección de portadores.

Finalmente, en la región correspondiente a la terminal \textit{gate}, se deposita un metal en contacto directo con el canal, formando una unión metal-semiconductor que permite controlar la conducción modulando la anchura de la región de agotamiento en el canal definida como $W_D$, esto se detalla en la figura \ref{fig:estr_Sze_MESFET}. Una diferencia crucial con respecto de los transistores de unión bipolar es que los transistores de efecto de campo no requieren de corriente de polarización y son controlados por tensión. Además, el hecho de que su funcionamiento responde a la corriente de portadores mayoritarios se los designa como transistores unipolares.

\begin{figure}[H]
    \centering

    \begin{minipage}{0.48\textwidth}
        \centering
        \includegraphics[width=\linewidth]{Img/neamen_structure.png}
        \caption{Estructura del transistor MESFET de GaAs \parencite{neamen2012}.}
       \label{fig:neamen_strucutre}
    \end{minipage}
    \hfill
    \begin{minipage}{0.48\textwidth}
        \centering
        \includegraphics[width=\linewidth]{Img/estructura_microwave.png}
        \caption{Estructura del transistor MESFET \parencite{giannini2009high}.}
       \label{fig:giannini_MESFET_structure}
    \end{minipage}

\end{figure}


En la figura \ref{fig:giannini_MESFET_structure} se detalla su estructura en tres dimensiones, también se detalla una capa opcional de material tipo p conocido como \textit{p-buffer} con el propósito de mejorar el acople, aislando mejor el canal, reduciendo corrientes de fuga hacia el sustrato y mejorando la estabilidad del dispositivo. Mientras que en la figura \ref{fig:simbologia_tipica} se detalla la simbología típica junto con referencias de tensión y corriente para esta clase de dispositivos.


\begin{figure}[H]
    \centering

    \begin{minipage}{0.45\textwidth}
        \centering
        \includegraphics[width=\linewidth]{Img/6W_MESFET_circuito_integrado.png}
         \caption{Imagen donde se observa un arreglo integrado de MESFET's de GaAs \parencite{giannini2009high} .}
    \label{fig:giannini_MESFETs}
    \includegraphics[width=\linewidth]{../Imagenes/MESFET_Simbolo}
         \caption{Simbología típica de transistor de efecto de campo.}
    \label{fig:simbologia_tipica}
    \end{minipage}
    \hfill
    \begin{minipage}{0.45\textwidth}
        \centering
        \includegraphics[width=\linewidth]{Img/Microscopio_MESFET_MISHRA.png}
         \caption{(Arriba) Una sección transversal de corte de un MESFET de 0,1 $\mu m$. (Abajo) Vista superior de la MESFET \parencite{mishra_semiconductor_2008}.}
    \label{fig:mishra_microscopio}
    \end{minipage}

\end{figure}




El dispositivo frente a una elevada temperatura disminuye su corriente evitando un descontrol térmico, esto permite conectar fácilmente varios MESFET en paralelo, creando así un dispositivo más grande, como se detalla, por ejemplo, en la figura \ref{fig:giannini_MESFETs}. Mientras que en la figura \ref{fig:mishra_microscopio} se aprecian dos imágenes reales del dispositivo mediante un microscopio, la primera mediante un corte trasversal y la segunda una visual superior.



\subsubsection{Características del Semiconductor y su Dopaje}

\begin{figure}[H]
	\centering
	\begin{minipage}{0.35\textwidth}
		\centering
		\includegraphics[width=\linewidth]
		{"../../../../Imágenes/Capturas de pantalla/zincblenda"}
		\caption{Estructura cristalina tipo zincblenda del semiconductor GaAs \parencite{neamen2012}.}
		\label{fig:zincblenda}
	\end{minipage}
	\hfill
	\begin{minipage}{0.55\textwidth}
		En el caso del Arseniuro de Galio, es un solido cuya red cristalina forma una estructura conocida como
		\textit{zincblenda} (Figura~\ref{fig:zincblenda}), compuesta por arsénico del grupo V y galio
		del grupo III. Al agregar una impureza, esta reemplaza a alguno de los átomos en la red
		cristalina. Para obtener material tipo N, la impureza donadora debe aportar un electrón
		adicional respecto del átomo que reemplaza. Cuando se emplea silicio como impureza, este
		sustituye al galio, que posee tres electrones de valencia, mientras que el silicio posee
		cuatro, actuando como donador. Por otro lado, el azufre y el selenio poseen seis electrones
		de valencia, por lo que al reemplazar al arsénico, que posee cinco, aportan un electrón
		adicional, dando lugar a material tipo N \parencite{neamen2012}.
		\vspace{1cm}
	\end{minipage}
\end{figure}


\subsection{Aplicaciones Típicas}

Es empleado en sistemas de comunicación por microondas, desde radiotelescopios hasta antenas parabólicas domésticas, sistemas satélitales y teléfonos celulares. En la actualidad los transistores MESFET han sido reemplazados en gran medida por transistores de alta movilidad de electrones (HEMT : High-electron-mobility transistors) y alternativas basadas en silicio Utilizando materiales diseñados para la banda prohibida. A pesar de esto, la tecnología MESFET sigue siendo relevante en aplicaciones especializadas de alta potencia y alta temperatura. Los dispositivos MESFET se suelen adoptar para frecuencias de hasta 18-20 GHz, mientras que la adopción de dispositivos de heterojunción (principalmente del tipo HEMT) se hace obligatoria para frecuencias de operación más altas \parencite{giannini2009high}.

\section{Principio de Funcionamiento}

Una unión metal–semiconductor puede dar lugar a dos tipos de contacto, dependiendo de la relación entre las funciones trabajo de los materiales que la conforman y del nivel de dopaje del semiconductor. El contacto óhmico, presente en los terminales \textit{drain} y \textit{source}, se obtiene utilizando un semiconductor tipo N fuertemente dopado, lo que permite una baja resistencia de contacto y condiciones cercanas a $\phi_m < \phi_{SC}$.

El segundo caso corresponde a la formación de una unión rectificante del tipo \textit{Schottky}, para la cual rige la relación $\phi_m > \phi_{SC}$. Esta situación se presenta cuando el semiconductor tipo N se encuentra levemente dopado y es la empleada en el terminal \textit{gate}. Este tipo de unión es también característico de los diodos rectificadores de rápida conmutación.

Al igual que en los diodos rectificadores polarizados en inversa, la unión Schottky forma, bajo determinada polarización, una región de vaciamiento de portadores libres en el semiconductor. En el transistor MESFET, la extensión de dicha región es modulada para permitir el estrangulamiento o la conducción del canal, siendo este el principio fundamental de su funcionamiento.


\subsection{Parámetros Característicos}

En el informe se tomara como objeto de estudio un transistor MESFET formado por GaAs, con silicio como impureza dopante en el canal y titanio como el metal de la compuerta.

Utilizando la siguiente notación, inspirada en la figura \ref{fig:sistcoordmesfetb}, tomando la referencia espacial $y = 0$ en la unión metal-semiconductor y asumiendo ciertas hipótesis, se definen los parámetros característicos del dispositivo:

\begin{figure}[H]
	\centering
	\begin{minipage}[t]{0.52\textwidth}
		\vspace{1.25 cm}
		\begin{itemize}
			\item $h:$ Espesor de la capa levemente dopada tipo N.
			\item $L:$ Longitud de la capa levemente dopada tipo N.
			\item $Z:$ Profundidad del dispositivo.
			\item $W_D(x):$ Espesor de la zona de vaciamiento, dependiente de $x$.
			\item $V(x) \in [V_S, V_{DS}]:$ Potencial eléctrico en la posición $x$.
			\item $N_D:$ Concentración de impurezas donadoras.
			\item $V_{bi}:$ Potencial de juntura.
			\item $q:$ Carga eléctrica del electrón.
			\item $\mu_n:$ Movilidad de electrones.
		\end{itemize}
	\end{minipage}
	\hfill
	\begin{minipage}[t]{0.43\textwidth}
		\vspace{0pt}
		\centering
		\includegraphics[width=\linewidth]{Img/3d_mishra_MESFET.png}
		\caption{Sistema de coordenadas propuesto.}
		\label{fig:sistcoordmesfetb}
		\vspace{0pt}
	\end{minipage}
\end{figure}
\texttt{Hipotesis [1] : ${N_D}^+ = N_D $ (impurezas totalmente ionizadas) y semiconductor no degenerado.}

\texttt{Hipotesis [2] : Solo hay carga en la región de vaciamiento.}



\subsubsection{Densidad de Carga}

La carga se formara en el semiconductor y en la superficie del metal, formando una región superficial con carga negativa y una región en el semiconductor de carga positiva de acuerdo a (\ref{equation:def_densidad_carga}) utilizando las hipótesis [1] y [2].

\begin{figure}[H]
	\centering
	
	% --------- Expresión analítica (izquierda) ----------
	\begin{minipage}[c]{0.45\textwidth}
		\centering
		\begin{equation}\label{equation:def_densidad_carga}
			\rho(x,y) =
			\begin{cases}
				0 & y < 0 \\[4pt]
				-q N_d W(x) & y = 0 \\[4pt]
				q N_d W(x) & 0 < y \le W(x) \\[4pt]
				0 & y > W(x)
			\end{cases}
		\end{equation}
	\end{minipage}
	\hfill
	% --------- Gráfico (derecha) ----------
	\begin{minipage}[c]{0.5\textwidth}
		\centering
		\begin{tikzpicture}[
			>=Latex,
			axis/.style={->, thick},
			charge/.style={fill=gray!40, draw=black},
			annot/.style={font=\small}
			]
			
			% Ejes
			\draw[axis] (-3,0) -- (3,0) node[right] {$y$};
			\draw[axis] (0,-3) -- (0,3) node[above] {$\rho(y)$};
			
			% Región positiva
			\draw[charge] (0,0) rectangle (2.5,2);
			\node[annot] at (1.25,1) {$qN_d W(x)$};
			
			% Región negativa (carga superficial)
			\draw[charge] (-0.1,-3) rectangle (0,0);
			\node[annot,left] at (-0.15,-1.3) {$-qN_d W(x)$};
			
			
			% Etiquetas de materiales
			\node[annot, anchor=east] at (-0.8,2.5) {Metal: Ti};
			\node[annot, anchor=west] at (0.8,2.5) {SC: GaAs tipo N};
			
			
			% W(x)
			\draw[<->] (0,-0.2) -- (2.5,-0.2)
			node[midway,below] {$W(x)$};
			
		\end{tikzpicture}
	\end{minipage}
	
	\caption{Expresión analítica y representación gráfica de la densidad de carga $\rho(x,y)$.}
\end{figure}



\subsubsection{Campo Eléctrico}

Para obtener el campo eléctrico (\ref{equation:campo_electrico}), utilizamos la ley de Gauss (\ref{equation:ley_gauss}):

\begin{equation}\label{equation:ley_gauss}
	\mathcal{E} = \frac{Q}{\epsilon_s}
\end{equation} 

\begin{figure}[h]
	\centering
	
	% --------- Expresión analítica (izquierda) ----------
	\begin{minipage}[c]{0.45\textwidth}
		\centering
		\begin{equation}\label{equation:campo_electrico}
			\mathcal{E}(x,y) =
			\begin{cases}
				0 & y \le 0 \\[4pt]
				\dfrac{q N_d}{\epsilon_s}\,[y - W(x)] 
				& 0 < y \le W(x) \\[6pt]
				0 & y > W(x)
			\end{cases}
		\end{equation}
	\end{minipage}
	\hfill
	% --------- Gráfico (derecha) ----------
	\begin{minipage}[c]{0.5\textwidth}
		\centering
		\begin{tikzpicture}[
			>=Latex,
			axis/.style={->, thick},
			annot/.style={font=\small}
			]
			
			% Ejes
			\draw[axis] (-3,0) -- (3,0) node[right] {$y$};
			\draw[axis] (0,-3) -- (0,2) node[above] {$\mathcal{E}(y)$};
			
			% Campo eléctrico en la región de vaciamiento (lineal)
			\draw[thick] (0,-2) -- (2.5,0);
			
			% Línea de referencia en y = W(x)
		%	\draw[dashed] (2.5,-2) -- (2.5,0);
			\node[annot, below] at (2.5,0.5) {$W(x)$};
			
			% Etiquetas de materiales (posiciones solicitadas)
			\node[annot, anchor=east] at (-0.8,1.2) {Metal: Ti};
				\node[annot, anchor=east] at (-0.1,-2) {$\mathcal{E}_{MAX}$};
			\node[annot, anchor=west] at (0.8,1.2) {SC: GaAs tipo N};
			
			% Pendiente
			\node[annot] at (1.8,-1.2)
			{$\displaystyle \frac{qN_d}{\epsilon_s}$};
			
		\end{tikzpicture}
	\end{minipage}
	
	\caption{Expresión analítica y representación gráfica del campo eléctrico $\mathcal{E}(x,y)$ en la unión metal--GaAs.}
\end{figure}

\subsubsection{Potencial Eléctrico}

Dada la relación entre el campo eléctrico y el gradiente del potencial eléctrico (\ref{equation:relacion_P_E}) se obtiene el potencial eléctrico, al integrar el potencial se debe tener en consideración una referencia inicial conveniente como $\phi (y = 0 ) = 0$ y teniendo en cuenta el potencial de juntura  $\phi [y = W(x) ] = \phi_{bi}$, analizado en mas profundidad en la sección de bandas de energía del informe.  

\begin{equation}\label{equation:relacion_P_E}
	\mathcal{E} = - \nabla \phi
\end{equation}

\begin{figure}[H]
	\centering
	
	% --------- Expresión analítica (izquierda) ----------
	\begin{minipage}[c]{0.45\textwidth}
		\centering
		\begin{equation}\label{equation:potencial_electrostatico}
			\phi(x,y) =
			\begin{cases}
				0 & y \le 0 \\[4pt]
				-\dfrac{q N_d}{2\epsilon_s}\,[y - W(x)]^2 + \phi_{bi}
				& 0 < y \le W(x) \\[6pt]
				\phi_{bi} & y > W(x)
			\end{cases}
		\end{equation}
	\end{minipage}
	\hfill
	% --------- Gráfico (derecha) ----------
	\begin{minipage}[c]{0.5\textwidth}
		\centering
		\begin{tikzpicture}[
			>=Latex,
			axis/.style={->, thick},
			annot/.style={font=\small},
			scale=1
			]
			
			% Ejes
			\draw[axis] (-3,0) -- (4,0) node[right] {$y$};
			\draw[axis] (0,-0.5) -- (0,2.5) node[above] {$\phi(y)$};
			
			% Potencial en la región de vaciamiento (parábola)
			\draw[thick, domain=0:3.5, samples=100]
			plot (\x, -{0.08*(\x-3.5)^2 + 1});
			
			% Línea constante para y > W(x)
			\draw[thick] (3,1) -- (3,1);
			\node[annot, right] at (3.3,1.3) {$\phi_{bi}$};
			
			% Línea vertical en W(x)
			\draw[dashed] (3,0) -- (3,1);
			\node[annot, below] at (3,0) {$W(x)$};
			
			% Etiquetas de materiales (posiciones solicitadas)
			\node[annot, anchor=east] at (-0.8,2) {Metal: Ti};
			\node[annot, anchor=west] at (0.8,2) {SC: GaAs tipo N};
			
			% Valor en y = 0
			\node[annot, below left] at (0,0) {$0$};
			
		\end{tikzpicture}
	\end{minipage}
	
	\caption{Expresión analítica y representación gráfica del potencial electrostático $\phi(x,y)$ en la unión metal--GaAs.}
\end{figure}


\section{Modos de Operación}
%modos de funcionamiento
En función de la tensión aplicada al los terminales \textit{gate} y  \textit{drain}, referidas al potencial del terminal \textit{source}, las condiciones eléctricas impuestas al dispositivo permiten definir tres regiones de operación.


\subsection{Régimen Corte}

Se da cuando $V_{GS} < V_T$, no se forma un canal considerable y el dispositivo no conduce corriente apreciable.

\subsection{Régimen de Estrangulación}


Se da cuando $V_{GS} > V_T$ y $V_{DS} > V_{DS_{[sat]}} $, el dispositivo conduce corriente de forma controlada en función de $V_{GS}$. En este modo de op


\subsection{Régimen Óhmico}

Se da cuando $V_{GS} > V_T$ y $V_{DS} < V_{DS_{[sat]}} $, el dispositivo comienza a comportarse como un resistor, esto ayuda a controlar los incrementos de temperatura.


\section{Diagrama de Bandas}

Para formar una unión metal-semiconductor, en condiciones ideales, se debe cumplir la relación de función trabajo entre el metal y el semiconductor $\phi_m > \phi_s$ los electrones fluyen desde el semiconductor hacia el metal formando la barrera de Schottky, en el caso contrario  $\phi_m < \phi_s$  formara una unión óhmica ya que el semiconductor estará fuertemente dopado y los electrones fluyen desde el metal hacia el semiconductor reduciendo o idealmente eliminando la barrera.

\begin{figure}[H]
	\centering
	\includegraphics[width=0.5\linewidth]{Img/bandas_energia_mishra.png}
	\caption{Esquema de un GaAs MESFET. También se muestra el perfil de banda de energía debajo de la
		Región de la puerta y algunos parámetros importantes del dispositivo \parencite{mishra_semiconductor_2008}.}
	\label{fig:placeholder}
\end{figure}


\section{Corriente de Salida}

\section{Comparación}

\section{Simulación de Curvas}

\subsection{Curva de Transferencia}
\subsection{Curva de Salida}

\section{Efectos No Ideales}

\subsection{Modulación del largo del Canal}
\subsection{Saturación de la velocidad de Arrastre }
\subsection{Región Subumbral}
\subsection{Efectos de corriente de Gate}

\section{Conclusión} 


Ejemplo referenicias\\
El comportamiento del MOSFET en saturación está bien modelado
por la ecuación cuadrática \parencite{sedra2015}.

Según \textcite{boylestad2013}, la tensión umbral depende del dopado.
El comportamiento de los portadores en semiconductores puede
modelarse a partir de principios cuánticos \parencite{neamen2012}.

Según \textcite{sze2021}, el MOSFET moderno requiere modelos
avanzados de canal corto.

El nivel de Fermi se introduce naturalmente desde la física
del estado sólido \parencite{mckelvey1996}.

\printbibliography

\section{Apéndice}

\subsection{Código}

\end{document}