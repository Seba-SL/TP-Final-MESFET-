\documentclass[a4paper]{article}
\usepackage[margin=1.5cm,top=1.5cm,bottom=1.5cm,a4paper]{geometry}
\usepackage{graphicx} % Required for inserting images
\usepackage{titlesec} % Required for customizing section titles
\usepackage[margin=1.5cm,top=1.5cm, bottom=1.5cm]{geometry} % Ajusta los márgenes aquí
\usepackage{multicol} % Required for multicols environment
\usepackage{parskip}
\usepackage{etoolbox}
\usepackage{float}
\usepackage{amsmath}
\usepackage{dingbat}
\usepackage{booktabs} 
\usepackage{caption}
\usepackage[utf8]{inputenc}
\usepackage[T1]{fontenc}
\usepackage{amssymb}
\usepackage[spanish]{babel}

\usepackage{tikz}
\usetikzlibrary{arrows.meta, positioning}
\usetikzlibrary{babel}

\usepackage{csquotes}

\usepackage[
  backend=biber,
  style=apa,
  citestyle=apa
]{biblatex}

\addbibresource{referencias.bib}



%\captionsetup[table]{name = Tabla}
 
% Redefinir el comando \thesection para usar números romanos
%\renewcommand{\thesection}{\Roman{section}}
% Redefinir el comando \thesubsection para usar números romanos
%\renewcommand{\thesubsection}{\thesection.\Roman{subsection}}


\renewcommand{\thesection}{\arabic{section}}
\renewcommand{\thesubsection}{\thesection.\arabic{subsection}}
\renewcommand{\thesubsubsection}{\thesubsection.\arabic{subsubsection}}


% Personalización de la sección para que comience desde la izquierda
\titleformat{\section}[hang]{\normalfont\Large\bfseries}{\thesection}{1em}{}

\begin{document}
\setcounter{page}{1}
\begin{minipage}[h]{1\textwidth}

\title{\includegraphics[height = 1.5cm]{logofiuba.png} % Logo en la izquierda del encabezado  
\\(TB070) Dispositivos Semiconductores \\Trabajo Práctico Final : Transistor MESFET }
\author{106213 Sebastián Lazo  ( slazo@fi.uba.ar ) }
\date{26 de Febrero 2026}

\maketitle

 \end{minipage}

\section{Resumen}

En el siguiente informe se utilizan los conocimientos adquiridos a lo largo de la materia Dispositivos Semiconductores con el objetivo de realizar una presentación, investigación y análisis en torno al dispositivo conocido como MESFET ( Metal-Semiconductor Field-Effect Transistor ). Se aborda su estructura, parámetros característicos y aplicaciones típicas, modos de operación. Ademas, se presentan diagramas de bandas para cada modo de operación, curvas características y se discute los efectos no ideales que influyen en su comportamiento real.

\section{Descripción del Dispositivo}

El transistor MESFET es un dispositivo multijuntura de tres terminales: fuente, drenaje y compuerta, cuya estructura forma un canal por el cual fluyen los portadores desde el terminal fuente hacia el terminal de drenaje. La conductividad del canal es modulada por el potencial eléctrico aplicado en el terminal de compuerta. Puede clasificarse como una variación del transistor MOSFET, con la diferencia de que no posee una capa de óxido aislante entre la compuerta y el sustrato, característica que también comparte con el transistor JFET. Con respecto a este último, difieren en el material de la compuerta, ya que en el MESFET no es de material semiconductor, sino metálica, empleándose metales como aluminio, titanio, oro, níquel o platino.

Para comprender la necesidad de esta clase de transistor es necesario conocer una limitación en los transistores MOSFET, estos pueden no ser apropiados si se desea fabricar un transistor utilizando ciertos materiales semiconductores, como en el caso del arseniuro de galio (GaAs), dado que en la interfase, entre el GaAs y un aislante como el oxido, se forman un gran número de trampas superficiales que inhiben la acción compuerta. Además, las compuertas de unión no se difunden con tanta facilidad dentro del GaAs debido a la inestabilidad del material a altas temperaturas, por lo tanto no se realizan uniones P-N con facilidad. Sin embargo, dado que el GaAs tiene una mayor movilidad que el silicio para sus portadores libres, su empleo es aconsejable en aplicaciones que requieren rápidas velocidades de conmutación. La mayoría de las aplicaciones utilizan el MESFET de canal N en lugar del de canal P debido a la mayor movilidad de las portadoras en los dispositivos de canal N \parencite{sze2002semiconductor}. Por consiguiente, se utilizan estructuras del tipo MESFET de canal N para cubrir esas necesidades \parencite{StreetmanBanerjee_SolidStateDevices_2006}. Otros semiconductores utilizados para fabricar MESFET's son nitruro de galio (GaN), carburo de silicio de estructura cristalina (4H-SiC) y fosforo de indio (InP), las caracteristicas de los mismos se encuentran en la tabla de la figura \ref{fig:tablapropsctransistores}. 


\begin{figure}[H]
	\centering
	\includegraphics[width=0.7\linewidth]{"../../../../Imágenes/Capturas de pantalla/TablaProp_SC_transistores"}
	\caption{Tabla con las caracteristicas princiáles de los semiconductores y el tipo de transistor fabricado con ellos \parencite{giannini2009high}.}
	\label{fig:tablapropsctransistores}
\end{figure}

\subsection{Primer Prototipo}

	El MESFET fue propuesto y demostrado por primera vez por Carver Mead en 1966, se puede observar una imagen de el primer prototipo en la figura \ref{fig:primermesfet}. Poco después, Hooper y Lehrer informaron sobre su rendimiento en dispositivos de microondas en 1967 \parencite{sze2021}. Carver Mead recibió una pequeña muestra de GaAs de tipo N formado epitaxialmente sobre una oblea semiaislante y comenzo a trabajar en el laboratorio grabando la capa epitaxial de su diminuta muestra de oblea, paso a paso, hasta que pudo ver la perforación desde un contacto metálico a una tension de 10 V. El proceso que estaba usando era un grabado químico de tiempo aproximado con metanol y bromo, y aunque 10 V era un poco alto, ya había pasado por varios pasos de grabado, metalización y medición (usando la relación entre capacitancia y tensión para obtener la concentración de dopaje del canal y espesor de agotamiento) para llegar a este punto. Decidió detenerse y, usando un rapidógrafo calentado (un bolígrafo de dibujo de tinta líquida de estilo antiguo) lleno de cera negra, dibujó una franja a lo largo de la oblea para definir el área activa de lo que sería la región de fuente-drenaje, y luego grabó cuidadosamente a través de la región epitaxial hasta el GaAs semiaislante en el resto. A continuación, e increíblemente, soldó por puntos dos hojas de afeitar de filo recto a pequeñas piezas transversales para hacer una máscara para una puerta muy estrecha. Los bordes de las hojas estaban tan juntos que formaban un patrón de difracción a lo largo de sus bordes casi en contacto. 

\begin{figure}[H]
	\centering
	
	\begin{minipage}{0.45\textwidth}	
		
		 Usando esta máscara improvisada, evaporó un electrodo de compuerta de aluminio muy fino (de unas pocas micras de ancho) a través del hueco entre las aspas y a lo largo del punto medio de la mesa. Después, soldó contactos óhmicos de indio-mercurio a la mesa, en lados opuestos de la compuerta, para formar la fuente y el drenador. Sorprendentemente, el dispositivo funcionó y Carver pudo registrar curvas I-V del FET, dado que el transistor operaba a 10 V. Redactó el breve artículo y lo presentó a las Actas del IEEE en diciembre, denominándolo Transistor de Efecto de Campo de Puerta de Barrera Schottky, que posteriormente se convertiría en el MESFET \parencite{Siegel2021Mead}.
		
		
	\end{minipage}
	\hfill
	\begin{minipage}{0.45\textwidth}
		\centering
		\includegraphics[width=\linewidth]{Img/primer_Mesfet}
		\caption{Fotografía del primer prototipo de MESFET de GaAs, tal como está conectado a un portaobjetos de microscopio \parencite{Siegel2021Mead}.}
		\label{fig:primermesfet}
	\end{minipage}
	
\end{figure}





\subsection{Estructura}

Los MESFET's se construyen a partir de una delgada capa epitaxial de GaAs de tipo N dopada con impurezas donadoras como pueden llegar a ser Silicio, Azufre o Selenio depositada sobre un sustrato semi-aislante tambien de GaAs pero dopado intencionalmente con Cromo, que se comporta como un único aceptor cerca del centro de la banda prohibida de energía, con el objetivo de obtener un semi-aislante con una resistividad de ordenes de hasta hasta $10^9 \frac{\Omega}{cm}$ \parencite{neamen2012}. Sobre esta capa se definen las tres terminales del dispositivo: \textit{source}, \textit{drain} y \textit{gate} (fuente, drenaje y compuerta), como se aprecia en la Figura \ref{fig:estr_Belove_MESFET}.



\begin{figure}[H]
    \centering

    \begin{minipage}{0.48\textwidth}
        \centering
        \includegraphics[width=\linewidth]{Img/estructura_MESFET.png}
        \caption{Estructura del transistor MESFET de GaAs \parencite{StreetmanBanerjee_SolidStateDevices_2006}.}
        \label{fig:estr_Belove_MESFET}
    \end{minipage}
    \hfill
    \begin{minipage}{0.48\textwidth}
        \centering
        \includegraphics[width=\linewidth]{Img/estructura_MESFET_SZE.png}
        \caption{Estructura del transistor MESFET, se observa la apertura neta del canal $b$ controlada por el ancho de agotamiento $W_D$ \parencite{sze2021}.}
        \label{fig:estr_Sze_MESFET}
    \end{minipage}

\end{figure}

Sobre el sustrato semi-aislante se encuentra la región activa del dispositivo, conformada por la capa de GaAs dopada ligeramente, en ella se forma el canal por el cual circulan los portadores mayoritarios cuando el dispositivo está en conducción, esto se aprecia en la figura \ref{fig:neamen_strucutre}.

En las zonas correspondientes a \textit{source} y \textit{drain}, esta misma capa se dopa fuertemente con las mismas impurezas de tipo N para obtener contactos óhmicos de baja resistencia, facilitando así la inyección y recolección de portadores.

Finalmente, en la región correspondiente a la terminal \textit{gate}, se deposita un metal en contacto directo con el canal, formando una unión metal-semiconductor que permite controlar la conducción modulando la anchura de la región de agotamiento en el canal definida como $W_D$, esto se detalla en la figura \ref{fig:estr_Sze_MESFET}. Una diferencia crucial con respecto de los transistores de unión bipolar es que los transistores de efecto de campo no requieren de corriente de polarización y son controlados por tensión. Además, el hecho de que su funcionamiento responde a la corriente de portadores mayoritarios se los designa como transistores unipolares.

\begin{figure}[H]
    \centering

    \begin{minipage}{0.48\textwidth}
        \centering
        \includegraphics[width=\linewidth]{Img/neamen_structure.png}
        \caption{Estructura del transistor MESFET de GaAs \parencite{neamen2012}.}
       \label{fig:neamen_strucutre}
    \end{minipage}
    \hfill
    \begin{minipage}{0.48\textwidth}
        \centering
        \includegraphics[width=\linewidth]{Img/estructura_microwave.png}
        \caption{Estructura del transistor MESFET \parencite{giannini2009high}.}
       \label{fig:giannini_MESFET_structure}
    \end{minipage}

\end{figure}





En la figura \ref{fig:giannini_MESFET_structure} se detalla su estructura en tres dimensiones, también se detalla una capa opcional de material tipo p conocido como \textit{p-buffer} con el propósito de mejorar el acople, aislando mejor el canal, reduciendo corrientes de fuga hacia el sustrato y mejorando la estabilidad del dispositivo.

\begin{figure}[H]
    \centering

    \begin{minipage}{0.45\textwidth}
        \centering
        \includegraphics[width=\linewidth]{Img/6W_MESFET_circuito_integrado.png}
         \caption{Imagen donde se observa un arreglo integrado de MESFET's de GaAs \parencite{giannini2009high} .}
    \label{fig:giannini_MESFETs}
   
    \includegraphics[width=\linewidth]{Img/esquematico_de_paper}
         \caption{ Esquema conceptual de polarización del MESFET \parencite{Belgat2004_MESFET_Interface}.}
    \label{fig:esquematico_de_paper}
    
	\end{minipage}
    \hfill
    \begin{minipage}{0.45\textwidth}
        \centering
        \includegraphics[width=\linewidth]{Img/Microscopio_MESFET_MISHRA.png}
         \caption{(Arriba) Una sección transversal de corte de un MESFET de 0,1 $\mu m$. (Abajo) Vista superior de la MESFET \parencite{mishra_semiconductor_2008}.}
    \label{fig:mishra_microscopio}
    \end{minipage}

\end{figure}


El dispositivo frente a una elevada temperatura disminuye su corriente evitando un descontrol térmico, esto permite conectar fácilmente varios MESFET en paralelo, creando así un dispositivo más grande, como se detalla, por ejemplo, en la figura \ref{fig:giannini_MESFETs}. Mientras que en la figura \ref{fig:mishra_microscopio} se aprecian dos imágenes reales del dispositivo mediante un microscopio, la primera mediante un corte trasversal y la segunda una visual superior.



\subsubsection{Características del Semiconductor y su Dopaje}

	En el caso del Arseniuro de Galio, es un solido cuya red cristalina forma una estructura conocida como
\textit{zincblenda} (Figura~\ref{fig:zincblenda}), compuesta por arsénico del grupo V y galio
del grupo III (Figura~\ref{fig:red_GaAs}). Al agregar una impureza, esta reemplaza a alguno de los átomos en la red
cristalina. Para obtener material tipo N, la impureza donadora debe aportar un electrón
adicional respecto del átomo que reemplaza. Cuando se emplea silicio como impureza, este
sustituye al galio, que posee tres electrones de valencia, mientras que el silicio posee
cuatro, actuando como donador. Por otro lado, el azufre y el selenio poseen seis electrones
de valencia, por lo que al reemplazar al arsénico, que posee cinco, aportan un electrón
adicional, dando lugar a material tipo N \parencite{neamen2012}.

\begin{figure}[H]
	\centering
	\begin{minipage}{0.35\textwidth}
		\centering
		\includegraphics[width=\linewidth]
		{"../../../../Imágenes/Capturas de pantalla/zincblenda"}
		\caption{Estructura cristalina tipo zincblenda del semiconductor GaAs \parencite{neamen2012}.}
		\label{fig:zincblenda}
	\end{minipage}
	\hfill
	\begin{minipage}{0.55\textwidth}
		\centering
		\includegraphics[width=0.5\linewidth]{Img/red_GaAs}
		\caption{Enlace covalente del cristal del GaAs. \parencite{BoylestadNashelsky2015}.}
		\label{fig:red_GaAs}
	\end{minipage}
\end{figure}


\subsection{Aplicaciones Típicas}
	
	
	Es empleado en sistemas de comunicación por microondas, amplificadores, radiotelescopios hasta antenas parabólicas domésticas, sistemas satélitales y teléfonos celulares operando en frecuencias por encima de 3 GHz. En la figura \ref{fig:gaas-mesfet-class-ab-rfpa-schematic-diagram} se implementa un MESFET en un amplificador clase AB.
	

\begin{figure}[H]
	\centering
	\begin{minipage}[t]{0.52\textwidth}
		\vspace{0.5 cm}
		 A pesar de esto, la tecnología MESFET sigue siendo relevante en aplicaciones especializadas de alta potencia y alta temperatura. Los dispositivos MESFET se suelen adoptar para frecuencias de hasta 18-20 GHz, mientras que la adopción de dispositivos de heterojunción (principalmente del tipo HEMT) se hace obligatoria para frecuencias de operación más altas \parencite{giannini2009high}.
		 
		 \vspace{1cm}
		 	
		 \begin{itemize}
		 	\item 	Radares
		 	\item 	Equipos de radiocomunicación
		 	\item 	Tacómetros
		 	\item 	Satélites
		 	\item 	Microondas
		 \end{itemize}
		 
	
	\end{minipage}
	\hfill
	\begin{minipage}[t]{0.43\textwidth}
		\vspace{0pt}
		\centering
		\includegraphics[width=1\linewidth]{../Imagenes/GaAs-MESFET-Class-AB-RFPA-schematic-diagram}
		\caption{Diagrama esquemático de un amplificador Class-AB RFPA implementando un GaAs MESFET.}
		\label{fig:gaas-mesfet-class-ab-rfpa-schematic-diagram}
		\vspace{0pt}
	\end{minipage}
\end{figure}



\section{Principio de Funcionamiento}

Una unión metal–semiconductor puede dar lugar a dos tipos de contacto, dependiendo de la relación entre las funciones trabajo de los materiales que la conforman y del nivel de dopaje del semiconductor. El contacto óhmico, presente en los terminales \textit{drain} y \textit{source}, se obtiene utilizando un semiconductor tipo N fuertemente dopado, lo que permite una baja resistencia de contacto y condiciones cercanas a $\phi_m < \phi_{SC}$.

El segundo caso corresponde a la formación de una unión rectificante del tipo \textit{Schottky}, para la cual rige la relación $\phi_m > \phi_{SC}$. Esta situación se presenta cuando el semiconductor tipo N se encuentra levemente dopado y es la empleada en el terminal \textit{gate}. Este tipo de unión es también característico de los diodos rectificadores de rápida conmutación.

Al igual que en los diodos Schottky polarizados en inversa, esta unión da lugar a la formación de una región de vaciamiento de portadores libres en el semiconductor. En el transistor MESFET, la extensión de dicha región es modulada mediante la tensión aplicada entre \textit{gate} y \textit{source}, permitiendo el estrangulamiento o la apertura del canal conductor. De esta forma, se controla la corriente que circula entre \textit{drain} y \textit{source}, constituyendo este mecanismo el principio fundamental de funcionamiento del dispositivo.


\subsection{Tipos de MESFET}

A la hora de fabricar estos transistores, se puede optar por un tipo de dopaje del canal tal que requiera tensiones umbrales negativas para el funcionamiento, el cual se conoce como modo empobrecimiento ( Depletion mode ) o tensiones de control positivas, este ultimo nombrado como modo enriquecimiento ( Enhancement mode).

\subsubsection{D-MESFET (Depletion-mode MESFET)}

\subsubsection{E-MESFET (Enhancement-mode MESFET}

\section{Parámetros Característicos}

En el informe se tomara como objeto de estudio un transistor MESFET formado por GaAs de canal n, con silicio como impureza dopante en el canal y titanio como el metal de la compuerta a una temperatura de 300 K. A continuación se presentaran los diferentes parámetros de referencia utilizados en el informe.


\subsection{Parámetros Físicos}

Los materiales que forman el dispositivo desprenden diferentes características descriptas por los siguientes parámetros (\ref{tab:param_fis_mesfet}). Tales parámetros determinan propiedades fundamentales del dispositivo, como la formación de la barrera \textit{Scottky}, la zona de vaciamiento, el transporte de portadores, etc.


\begin{table}[h]
	\centering
	\caption{Parámetros físicos característicos de los materiales utilizados en el análisis del MESFET}
	\label{tab:param_fis_mesfet}
	\begin{tabular}{|l|c|c|c|c|}
		\hline
		\textbf{Parámetro} & \textbf{GaAs} & \textbf{Si (dopante)} & \textbf{Ti} & \textbf{Cr} \\ \hline
		Tipo de material & Semiconductor & Semiconductor & Metal & Metal \\ \hline
		Energía de banda prohibida $E_g$ [eV] & 1.42 & 1.12 & -- & -- \\ \hline
		Constante dieléctrica relativa $\epsilon_r$ & 12.9 & 11.7 & -- & -- \\ \hline
		Afinidad electrónica $\chi$ [eV] & 4.07 & -- & -- & -- \\ \hline
		Función trabajo $\phi_m$ [eV] & -- & -- & 4.33 & 4.50 \\ \hline
		Movilidad electrónica $\mu_n$ [cm$^2$/Vs] & 8500 & -- & -- & -- \\ \hline
		Concentración  $N_D$ [cm$^{-3}$] & -- & $N_D$ [cm$^{-3}$] & -- & -- \\ \hline
		Tipo de contacto en MESFET & Canal & Dopaje & Schottky / Óhmico & Aislante \\ \hline
	\end{tabular}
\end{table}

\subsection{Parámetros Geométricos}

En la siguiente tabla se presentan los parámetros que describen las dimensiones geométricas del dispositivo (\ref{tab:param_geom_mesfet}). 

\begin{table}[H]
	\centering
	\caption{Parámetros geométricos característicos utilizados en el análisis del MESFET}
	\label{tab:param_geom_mesfet}
	\begin{tabular}{|l|c|l|}
		\hline
		\textbf{Parámetro} & \textbf{Símbolo} & \textbf{Descripción} \\ \hline
		Longitud de la compuerta & $L$ & Longitud del canal en la dirección $x$ (source--drain)   \\ \hline
		Ancho del dispositivo & $Z$ & Dimensión del canal en la dirección $z$ (profundidad del dispositivo) \\ \hline
		Espesor del canal & $d$ & Altura del canal en la dirección $y$ \\ \hline
		Ancho de la zona de vaciamiento & $W_d$ & Extensión de la región de vaciamiento bajo la compuerta \\ \hline
		Espesor efectivo del canal & $d_{eff}$ & Espesor del canal conductor: $d_{eff}=d-W_d$ \\ \hline
		Área efectiva del canal & $A_{ch}$ & Área transversal del canal: $A_{ch}=Z\cdot d_{eff}$ \\ \hline
		Posición longitudinal & $x$ & Coordenada a lo largo del canal \\ \hline
		Posición vertical & $y$ & Coordenada perpendicular a la superficie \\ \hline
	\end{tabular}
\end{table}


\subsection{Parámetros Eléctricos}

El comportamiento eléctrico del dispositivo se podrá describir con ayuda de las definiciónes de diferentes tensiones, corrientes, resistencias, capacitancias y otros parámetros eléctricos para facilitar el análisis del dispositivo (\ref{tab:param_elec_mesfet}). 

\begin{table}[H]
	\centering
	\caption{Parámetros eléctricos característicos utilizados en el análisis del MESFET}
	\label{tab:param_elec_mesfet}
	\begin{tabular}{|l|c|c|l|}
		\hline
		\textbf{Parámetro} & \textbf{Símbolo} & \textbf{Unidad} & \textbf{Descripción} \\ \hline
		Tensión compuerta--fuente & $V_{GS}$ & V & Tensión aplicada a la unión Schottky de la compuerta \\ \hline
		Tensión drenaje--fuente & $V_{DS}$ & V & Tensión aplicada entre drenaje y fuente \\ \hline
		Corriente de drenaje & $I_D$ & A & Corriente que circula por el canal del MESFET \\ \hline
		Corriente de compuerta & $I_G$ & A & Corriente de fuga de la unión Schottky \\ \hline
		Tensión de umbral (pinch--off) & $V_T$ & V & Tensión $V_{GS}$ para la cual el canal se estrangula \\ \hline
		Corriente de saturación & $I_{DSS}$ & A & Corriente de drenaje para $V_{GS} = 0$ (D-MESFET)\\ \hline
		Transconductancia & $g_m$ & $ \Omega^{-1}$ & Variación de $I_D$ respecto de $V_{GS}$ \\ \hline
		Resistencia de salida & $r_o$ & $\Omega$ & Inversa de la pendiente $I_D$--$V_{DS}$ en saturación \\ \hline
		Capacitancia compuerta--fuente & $C_{GS}$ & F & Capacitancia de la región de vaciamiento bajo la compuerta \\ \hline
		Capacitancia compuerta--drenaje & $C_{GD}$ & F & Capacitancia parásita asociada al solapamiento \\ \hline
		Campo eléctrico crítico & $xi_{crit}$ & V/cm & Campo máximo antes de ruptura del GaAs \\ \hline
	\end{tabular}
\end{table}


\subsection{Parámetros de Rendimiento}

Para un posterior análisis de funcionamiento detallado contemplando posibles condiciones no ideales, es necesario detallar parámetros de rendimiento como los descriptos por la siguiente tabla (\ref{tab:param_rend_mesfet}). 



\begin{table}[H]
	\centering
	\caption{Parámetros de rendimiento característicos del transistor MESFET}
	\label{tab:param_rend_mesfet}
	\begin{tabular}{|l|c|l|}
		\hline
		\textbf{Parámetro} & \textbf{Símbolo} & \textbf{Descripción} \\ \hline
		Frecuencia de transición & $f_T$ & Frecuencia para la cual la ganancia de corriente es unitaria \\ \hline
		Frecuencia máxima de oscilación & $f_{max}$ & Frecuencia límite para ganancia de potencia unitaria \\ \hline
		Ancho de banda útil & $BW$ & Rango de frecuencias de operación efectiva \\ \hline
		Factor de ruido & $NF$ & Degradación de la relación señal--ruido introducida por el dispositivo \\ \hline
		Temperatura de operación & $T_{op}$ & Rango de temperatura en condiciones nominales \\ \hline
		Rango de tensión compuerta--fuente & $V_{GS}$ & Intervalo de polarización segura de la compuerta \\ \hline
		Rango de tensión drenaje--fuente & $V_{DS}$ & Intervalo de operación sin ruptura \\ \hline
		Rango de corriente de drenaje & $I_D$ & Corriente admisible en régimen continuo \\ \hline
		Región de operación recomendada & -- & Lineal, saturación o corte \\ \hline
		Linealidad & IP3, $P_{1\text{dB}}$ & Capacidad de mantener proporcionalidad señal--salida \\ \hline
		Potencia disipada máxima & $P_{max}$ & Potencia máxima admisible sin degradación \\ \hline
		Estabilidad & $K$ & Factor de estabilidad (Rollet) \\ \hline
	\end{tabular}
\end{table}



	%	\includegraphics[width=\linewidth]{Img/3d_mishra_MESFET.png}
	%	\caption{Sistema de coordenadas propuesto \parencite{mishra_semiconductor_2008}.}
	%	\label{fig:sistcoordmesfetb}
	


\section{Operación Básica}

\begin{figure}[H]
	\centering
	\includegraphics[width=0.7\linewidth]{Img/simbolo_mas_esquema}
	\caption{Simbología tipica y esquema \parencite{Sharma_Metal_Semiconductor}. }
	\label{fig:simbolomasesquema}
\end{figure}





\subsection{Polarización}


\subsection{Modo Corte}
\subsection{Modo en Estrangulación}
\subsection{Modo Óhmico}

\section{Diagrama de Bandas}


\subsection{Modo Corte}
\subsection{Modo en Estrangulación}
\subsection{Modo Óhmico}

\section{Obtención del Modelo Completo}

Utilizando el sistema de coordenadas y notación inspirados en la figura \ref{fig:sistcoordmesfetb}, tomando la referencia espacial $y = 0$ en la unión metal-semiconductor y asumiendo ciertas hipótesis, se puede hallar la expresión de la corriente partiendo :

\texttt{Hipotesis [1] : ${N_D}^+ = N_D $ (impurezas totalmente ionizadas) y semiconductor no degenerado.}

\texttt{Hipotesis [2] : Solo hay carga en la región de vaciamiento.}


 Los electrones libres del semiconductor en las inmediaciones de la unión se transfieren hacia la superficie del metal, formando una carga superficial negativa. En el semiconductor se establece una zona de átomos donadores ionizados que conforman una carga positiva. Este proceso al alcanzar el equilibrio termodinámico (ETD) permite determinar, en reposo, la densidad de carga, el campo electrostático que se forma y el potencial eléctrico.


\subsection{Densidad de Carga}

La carga se formara en el semiconductor y en la superficie del metal, formando una región superficial con carga negativa y una región en el semiconductor de carga positiva de acuerdo a (\ref{equation:def_densidad_carga}) utilizando las hipótesis [1] y [2].

\begin{figure}[H]
	\centering
	
	% --------- Expresión analítica (izquierda) ----------
	\begin{minipage}[c]{0.45\textwidth}
		\centering
		\begin{equation}\label{equation:def_densidad_carga}
			\rho(x,y) =
			\begin{cases}
				0 & y < 0 \\[4pt]
				-q N_d W(x) & y = 0 \\[4pt]
				q N_d W(x) & 0 < y \le W(x) \\[4pt]
				0 & y > W(x)
			\end{cases}
		\end{equation}
	\end{minipage}
	\hfill
	% --------- Gráfico (derecha) ----------
	\begin{minipage}[c]{0.5\textwidth}
		\centering
		\begin{tikzpicture}[
			>=Latex,
			axis/.style={->, thick},
			charge/.style={fill=gray!40, draw=black},
			annot/.style={font=\small}
			]
			
			% Ejes
			\draw[axis] (-3,0) -- (3,0) node[right] {$y$};
			\draw[axis] (0,-3) -- (0,3) node[above] {$\rho(y)$};
			
			% Región positiva
			\draw[charge] (0,0) rectangle (2.5,2);
			\node[annot] at (1.25,1) {$qN_d W(x)$};
			
			% Región negativa (carga superficial)
			\draw[charge] (-0.1,-3) rectangle (0,0);
			\node[annot,left] at (-0.15,-1.3) {$-qN_d W(x)$};
			
			
			% Etiquetas de materiales
			\node[annot, anchor=east] at (-0.8,2.5) {Metal: Ti};
			\node[annot, anchor=west] at (0.8,2.5) {SC: GaAs tipo N};
			
			
			% W(x)
			\draw[<->] (0,-0.2) -- (2.5,-0.2)
			node[midway,below] {$W(x)$};
			
		\end{tikzpicture}
	\end{minipage}
	
	\caption{Expresión analítica y representación gráfica de la densidad de carga $\rho(x,y)$.}
\end{figure}



\subsection{Campo Eléctrico}

Para obtener el campo eléctrico (\ref{equation:campo_electrico}), utilizamos la ley de Gauss (\ref{equation:ley_gauss}):

\begin{equation}\label{equation:ley_gauss}
	\mathcal{E} = \frac{Q}{\epsilon_s}
\end{equation} 

\begin{figure}[h]
	\centering
	
	% --------- Expresión analítica (izquierda) ----------
	\begin{minipage}[c]{0.45\textwidth}
		\centering
		\begin{equation}\label{equation:campo_electrico}
			\mathcal{E}(x,y) =
			\begin{cases}
				0 & y \le 0 \\[4pt]
				\dfrac{q N_d}{\epsilon_s}\,[y - W(x)] 
				& 0 < y \le W(x) \\[6pt]
				0 & y > W(x)
			\end{cases}
		\end{equation}
	\end{minipage}
	\hfill
	% --------- Gráfico (derecha) ----------
	\begin{minipage}[c]{0.5\textwidth}
		\centering
		\begin{tikzpicture}[
			>=Latex,
			axis/.style={->, thick},
			annot/.style={font=\small}
			]
			
			% Ejes
			\draw[axis] (-3,0) -- (3,0) node[right] {$y$};
			\draw[axis] (0,-3) -- (0,2) node[above] {$\mathcal{E}(y)$};
			
			% Campo eléctrico en la región de vaciamiento (lineal)
			\draw[thick] (0,-2) -- (2.5,0);
			
			% Línea de referencia en y = W(x)
		%	\draw[dashed] (2.5,-2) -- (2.5,0);
			\node[annot, below] at (2.5,0.5) {$W(x)$};
			
			% Etiquetas de materiales (posiciones solicitadas)
			\node[annot, anchor=east] at (-0.8,1.2) {Metal: Ti};
				\node[annot, anchor=east] at (-0.1,-2) {$\mathcal{E}_{MAX}$};
			\node[annot, anchor=west] at (0.8,1.2) {SC: GaAs tipo N};
			
			% Pendiente
			\node[annot] at (1.8,-1.2)
			{$\displaystyle \frac{qN_d}{\epsilon_s}$};
			
		\end{tikzpicture}
	\end{minipage}
	
	\caption{Expresión analítica y representación gráfica del campo eléctrico $\mathcal{E}(x,y)$ en la unión metal--GaAs.}
\end{figure}

\subsection{Potencial Eléctrico}

Dada la relación entre el campo eléctrico y el gradiente del potencial eléctrico (\ref{equation:relacion_P_E}) se obtiene el potencial eléctrico, al integrar el potencial se debe tener en consideración una referencia inicial conveniente como $\phi (y = 0 ) = 0$ y inicialmente teniendo solo en consideración el potencial de juntura $\phi [y = W(x) ] = \phi_{bi}$, analizado en mas profundidad en la sección de bandas de energía del informe. 

\begin{equation}\label{equation:relacion_P_E}
	\mathcal{E} = - \nabla \phi
\end{equation}

\begin{figure}[H]
	\centering
	
	% --------- Expresión analítica (izquierda) ----------
	\begin{minipage}[c]{0.45\textwidth}
		\centering
		\begin{equation}\label{equation:potencial_electrostatico}
			\phi(x,y) =
			\begin{cases}
				0 & y \le 0 \\[4pt]
				-\dfrac{q N_d}{2\epsilon_s}\,[y - W(x)]^2 + \phi_{bi}
				& 0 < y \le W(x) \\[6pt]
				\phi_{bi} & y > W(x)
			\end{cases}
		\end{equation}
	\end{minipage}
	\hfill
	% --------- Gráfico (derecha) ----------
	\begin{minipage}[c]{0.5\textwidth}
		\centering
		\begin{tikzpicture}[
			>=Latex,
			axis/.style={->, thick},
			annot/.style={font=\small},
			scale=1
			]
			
			% Ejes
			\draw[axis] (-3,0) -- (4,0) node[right] {$y$};
			\draw[axis] (0,-0.5) -- (0,2.5) node[above] {$\phi(y)$};
			
			% Potencial en la región de vaciamiento (parábola)
			\draw[thick, domain=0:3.5, samples=100]
			plot (\x, -{0.08*(\x-3.5)^2 + 1});
			
			% Línea constante para y > W(x)
			\draw[thick] (3,1) -- (3,1);
			\node[annot, right] at (3.3,1.3) {$\phi_{bi}$};
			
			% Línea vertical en W(x)
			\draw[dashed] (3,0) -- (3,1);
			\node[annot, below] at (3,0) {$W(x)$};
			
			% Etiquetas de materiales (posiciones solicitadas)
			\node[annot, anchor=east] at (-0.8,2) {Metal: Ti};
			\node[annot, anchor=west] at (0.8,2) {SC: GaAs tipo N};
			
			% Valor en y = 0
			\node[annot, below left] at (0,0) {$0$};
			
		\end{tikzpicture}
	\end{minipage}
	
	\caption{Expresión analítica y representación gráfica del potencial electrostático $\phi(x,y)$ en la unión metal--GaAs.}
\end{figure}

\subsection{Ancho de Zona de Vaciamiento}

Relacionando el campo eléctrico (\ref{equation:campo_electrico}) junto con el potencial (\ref{equation:potencial_electrostatico}) se puede despejar el ancho de la zona de agotamiento o vaciamiento en reposo y equilibrio térmico.

\begin{equation}\label{equation: W_d_reposo}
	W_d = \sqrt{\frac{2 \cdot \epsilon_s \cdot \phi_{bi}}{q\cdot N_d}}
\end{equation} 


Al considerar la polarización del dispositivo se produce un potencial longitudinal $V(x)$ y el potencial externo trasversal al canal entre \textit{gate} y \textit{source} que afectaran a la modulación de la zona de vaciamiento.

\begin{equation}\label{equation: W_d_polarizado}
	W_d(x) = \sqrt{\frac{2 \cdot \epsilon_s \cdot [  \phi_{bi} - V_{GS} - V(x)] }{q\cdot N_d}}
\end{equation} 


\section{Corriente}

Para determinar la corriente en un dispositivo semiconductor se consideran dos fenómenos de transporte, arrastre y difusión de los portadores.




\section{Modos de Operación}
%modos de funcionamiento
En función de la tensión aplicada al los terminales \textit{gate} y  \textit{drain}, referidas al potencial del terminal \textit{source}, las condiciones eléctricas impuestas al dispositivo permiten definir tres regiones de operación.


\subsection{Régimen Corte}

Se da cuando $V_{GS} < V_T$, no se forma un canal considerable y el dispositivo no conduce corriente apreciable.

\subsection{Régimen de Estrangulación}


Se da cuando $V_{GS} > V_T$ y $V_{DS} > V_{DS_{[sat]}} $, el dispositivo conduce corriente de forma controlada en función de $V_{GS}$. En este modo de op


\subsection{Régimen Óhmico}

Se da cuando $V_{GS} > V_T$ y $V_{DS} < V_{DS_{[sat]}} $, el dispositivo comienza a comportarse como un resistor, esto ayuda a controlar los incrementos de temperatura.


\section{Diagrama de Bandas}

Para formar una unión metal-semiconductor, en condiciones ideales, se debe cumplir la relación de función trabajo entre el metal y el semiconductor $\phi_m > \phi_s$ los electrones fluyen desde el semiconductor hacia el metal formando la barrera de Schottky, en el caso contrario  $\phi_m < \phi_s$  formara una unión óhmica ya que el semiconductor estará fuertemente dopado y los electrones fluyen desde el metal hacia el semiconductor reduciendo o idealmente eliminando la barrera.

\begin{figure}[H]
	\centering
	\includegraphics[width=0.5\linewidth]{Img/bandas_energia_mishra.png}
	\caption{Esquema de un GaAs MESFET. También se muestra el perfil de banda de energía debajo de la
		Región de la puerta y algunos parámetros importantes del dispositivo \parencite{mishra_semiconductor_2008}.}
	\label{fig:placeholder}
\end{figure}


\section{Corriente de Salida}

\section{Comparación}

\section{Simulación de Curvas}

\subsection{Curva de Transferencia}
\subsection{Curva de Salida}

\section{Efectos No Ideales}

\subsection{Modulación del largo del Canal}
\subsection{Saturación de la velocidad de Arrastre }
\subsection{Región Subumbral}
\subsection{Efectos de corriente de Gate}

\section{Conclusión} 


Ejemplo referenicias\\
El comportamiento del MOSFET en saturación está bien modelado
por la ecuación cuadrática \parencite{sedra2015}.

Según \textcite{boylestad2013}, la tensión umbral depende del dopado.
El comportamiento de los portadores en semiconductores puede
modelarse a partir de principios cuánticos \parencite{neamen2012}.

Según \textcite{sze2021}, el MOSFET moderno requiere modelos
avanzados de canal corto.

El nivel de Fermi se introduce naturalmente desde la física
del estado sólido \parencite{mckelvey1996}.

\printbibliography

\section{Apéndice}

\subsection{Código}

\end{document}